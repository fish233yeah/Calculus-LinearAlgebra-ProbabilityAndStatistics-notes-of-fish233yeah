\documentclass[12pt,a4paper,UTF8]{book}
\usepackage{geometry}   %页边距调整
\geometry{left=2.0cm,right=2.0cm,top=2.5cm,bottom=2.5cm}
\usepackage[hyperref, UTF8]{ctex}   %中文输入,及生成PDF时自动生成书签
\hypersetup{colorlinks=true,linkcolor=black}   %取消目录的默认红框显示
\usepackage{makeidx}
\usepackage{pgf}   %创建PostScript和pdf格式的图形
\usepackage{tikz}   %配合pgf宏包创建复杂图表,例如函数图像
\usetikzlibrary{math}   %tikz宏包中的库,用于在画图时自己创建变量
\usepackage{extpfeil}   %extpfeil扩展添加了更多用于生成可扩展箭头的宏,例如\xlongequal等
\usepackage{txfonts}   %数学字符库
\usepackage{amsmath}   %更好的数学排版环境,里面提供了更多函数,例如\text{}、\operatorname{}等
\usepackage{latexsym}   %使用不同的符号字体宏包
\usepackage{longtable}   %可换页表格
\usepackage{graphicx}   %可导入和调整外部图片
\usepackage{enumerate}   %自定义编号标签

\begin{document}
\frontmatter
\title{高等数学笔记}
\author{衔瑜\thanks{Email: fish233yeah@163.com}}
\date{\today}
\maketitle
\tableofcontents



\mainmatter
\setlength{\parindent}{0pt}
\chapter{知识点总结}

\section{函数与极限}
\subsection{数列求和}
\begin{enumerate}
\item 常用裂项公式
\begin{enumerate}
\item $\dfrac{n}{\left(n+1\right)!}=\dfrac{\left(n+1\right)-1}{\left(n+1\right)!}=\dfrac{1}{n!}-\dfrac{1}{\left(n+1\right)!}$
\item $\dfrac{1}{k\left(k+1\right)\left(k+2\right)}=\dfrac{1}{2}\left[\dfrac{1}{k\left(k+1\right)}-\dfrac{1}{\left(k+1\right)\left(k+2\right)}\right]$
\end{enumerate}
\end{enumerate}

\subsection{三角函数}
\begin{enumerate}
\item 二角和差公式
\begin{enumerate}
\item $\sin\left(\alpha+\beta\right)=\sin\alpha\cos\beta+\cos\alpha\sin\beta$
\item $\cos\left(\alpha+\beta\right)=\cos\alpha\cos\beta-\sin\alpha\sin\beta$
\item $\tan\left(\alpha+\beta\right)=\dfrac{\tan\alpha+\tan\beta}{1-\tan\alpha\tan\beta}$
\end{enumerate}
\item 二倍角公式
\begin{enumerate}
\item $\sin2\alpha=2\sin\alpha\cos\alpha$
\item $\cos2\alpha=\cos^2\alpha-\sin^2\alpha$
\end{enumerate}
\item 和差化积公式
\begin{enumerate}
\item $\sin\alpha+\sin\beta=2\sin\dfrac{\alpha+\beta}{2}\cos\dfrac{\alpha-\beta}{2}$
\item $\sin\alpha-\sin\beta=2\cos\dfrac{\alpha+\beta}{2}\sin\dfrac{\alpha-\beta}{2}$
\item $\cos\alpha+\cos\beta=2\cos\dfrac{\alpha+\beta}{2}\cos\dfrac{\alpha-\beta}{2}$
\item $\cos\alpha-\cos\beta=-2\sin\dfrac{\alpha+\beta}{2}\sin\dfrac{\alpha-\beta}{2}$
\item $\tan\alpha+\tan\beta=\dfrac{\sin\left(\alpha+\beta\right)}{\cos\alpha\cos\beta}$
\end{enumerate}
\item 积化和差公式
\begin{enumerate}
\item $\cos\alpha\sin\beta=\dfrac{1}{2}\left[\sin\left(\alpha+\beta\right)-\sin\left(\alpha-\beta\right)\right]$
\item $\cos\alpha\cos\beta=\dfrac{1}{2}\left[\cos\left(\alpha-\beta\right)+\cos\left(\alpha+\beta\right)\right]$
\item $\sin\alpha\sin\beta=\dfrac{1}{2}\left[\cos\left(\alpha-\beta\right)-\cos\left(\alpha+\beta\right)\right]$
\end{enumerate}
\item 辅助角公式
\begin{enumerate}
\item $a\sin\alpha+b\cos\beta=\sqrt{a^2+b^2}\sin\left(\alpha+\varphi\right),\quad\text{此时}\sin\varphi=\dfrac{b}{\sqrt{a^2+b^2}},\cos\varphi=\dfrac{a}{\sqrt{a^2+b^2}}$
\end{enumerate}
\end{enumerate}

\subsection{双曲函数与反双曲函数}
\begin{enumerate}
\item 双曲函数
\begin{enumerate}
\item 双曲正弦:$\sinh x=\dfrac{e^x-e^{-x}}{2}$
\item 双曲余弦:$\cosh x=\dfrac{e^x+e^{-x}}{2}$
\item 双曲正切:$\tanh x=\dfrac{\sinh x}{\cosh x}=\dfrac{e^x-e^{-x}}{e^x+e^{-x}}$
\end{enumerate}
\item 反双曲函数
\begin{enumerate}
\item 反双曲正弦:$y=\operatorname{arsh}x=\ln\left(x+\sqrt{x^2+1}\right)$
\item 反双曲余弦:$y=\operatorname{arch}x=\ln\left(x+\sqrt{x^2-1}\right)$
\item 反双曲正切:$y=\operatorname{arth}x=\dfrac{1}{2}\ln\left(\dfrac{1+x}{1-x}\right)$
\end{enumerate}
\item 双曲函数运算公式
\begin{enumerate}
\item $\cosh^2x-\sinh^2x=1$
\item $\sinh\left(x+y\right)=\sinh x\cosh y+\cosh x\sinh y$
\item $\sinh\left(x-y\right)=\sinh x\cosh y-\cosh x\sinh y$
\item $\cosh\left(x+y\right)=\cosh x\cosh y+\sinh x\sinh y$
\item $\cosh\left(x-y\right)=\cosh x\cosh y-\sinh x\sinh y$
\end{enumerate}
\end{enumerate}

\subsection{函数间断点分类}
\begin{enumerate}
\item \textbf{第一类间断点}:左极限和右极限均存在的间断点。包括可去间断点、跳跃间断点
\item \textbf{第二类间断点}:非第一类间断点的任意间断点。包括无穷间断点、振荡间断点
\end{enumerate}

\subsection{等价无穷小}
\begin{enumerate}
\item 一阶:
\begin{enumerate}
\item $x\sim\sin x\sim\arcsin x$
\item $x\sim\tan x\sim\arctan x$
\item $x\sim\ln\left(1+x\right)$
\item $x\sim\ln a\cdot\log_a\left(1+x\right)$
\item $x\sim e^x-1$
\item $x\sim\dfrac{1}{\ln a}\left(a^x-1\right)$
\item $\lambda x\sim\left(1+x\right)^\lambda-1$
\end{enumerate}
\item 二阶:
\begin{enumerate}
\item $\dfrac{x^2}{2}\sim 1-\cos x$
\item $\dfrac{x^2}{2}\sim\sec x-1$
\item $\dfrac{x^2}{2}\sim x-\ln\left(1+x\right)$
\item $\dfrac{x^2}{2}\sim e^x-x-1$
\end{enumerate}
\item 三阶:
\begin{enumerate}
\item $\dfrac{x^3}{2}\sim\tan x-\sin x$
\item $\dfrac{x^3}{6}\sim x-\sin x$
\item $\dfrac{x^3}{3}\sim\tan x-x$
\item $\dfrac{x^3}{6}\sim\arcsin x-x$
\item $\dfrac{x^3}{3}\sim x-\arctan x$
\end{enumerate}
\end{enumerate}


\section{导数与微分}
\subsection{反函数求导}
\begin{enumerate}
\item 对于函数$y=f\left(x\right)$,其反函数的导数即为其导数的倒数,即$f^{-1}\left(y\right)=\dfrac{dx}{dy}=\dfrac{1}{\dfrac{dy}{dx}}$
\end{enumerate}

\subsection{洛必达法则}
\begin{enumerate}
\item 若函数$f\left(x\right)$与$g\left(x\right)$满足
\begin{enumerate}
\item $\lim\limits_{x\to a}f\left(x\right)=\lim\limits_{x\to a}g\left(x\right)=0\text{(或者}\infty\text{)}$
\item $f\left(x\right)$与$g\left(x\right)$在点$a$的某去心邻域内都可导,且$g^{\prime}\left(a\right)\neq0$
\item $\lim\limits_{x\to a}\dfrac{f^{\prime}\left(x\right)}{g^{\prime}\left(x\right)}=A\text{(}A\text{可以为实数也可以为}\pm\infty\text{)}$
\end{enumerate}
则
\[\lim\limits_{x\to a}\frac{f\left(x\right)}{g\left(x\right)}=\lim\limits_{x\to a}\frac{f^{\prime}\left(x\right)}{g^{\prime}\left(x\right)}=A\]
\end{enumerate}

\subsection{常用导数公式}
\begin{enumerate}
\item $\left(a^x\right)^\prime=a^x\ln a$
\item $\left(\log_ax\right)^\prime=\dfrac{1}{x\ln a}$
\item $\left(\arcsin x\right)^\prime=\dfrac{1}{\sqrt{1-x^2}}$
\item $\left(\arccos x\right)^\prime=-\dfrac{1}{\sqrt{1-x^2}}$
\item $\left(\arctan x\right)^\prime=\dfrac{1}{1+x^2}$
\item $\left(\operatorname{arccot}x\right)^\prime=-\dfrac{1}{1+x^2}$
\item $\left(\sinh x\right)^\prime=\cosh x$
\item $\left(\cosh x\right)^\prime=\sinh x$
\item $\left(\tanh x\right)^\prime=\dfrac{1}{\cosh^2x}$
\item $\left(\operatorname{arsh}x\right)^\prime=\dfrac{1}{\sqrt{1+x^2}}$
\item $\left(\operatorname{arch}x\right)^\prime=\dfrac{1}{\sqrt{x^2-1}}$
\item $\left(\operatorname{arth}x\right)^\prime=\dfrac{1}{1-x^2}$
\end{enumerate}

\subsection{微分公式}
\begin{enumerate}
\item $d\left(u\pm v\right)=du\pm dv$
\item $d\left(Cu\right)=Cdu$
\item $d\left(uv\right)=vdu+udv$
\item $d\left(\dfrac{u}{v}\right)=\dfrac{vdu-udv}{v^2}$
\end{enumerate}


\section{微分中值定理}
\subsection{三大中值定理}
\begin{enumerate}
\item \textbf{罗尔定理}:若$f\left(x\right)$在$\left[a,b\right]$上连续且在$\left(a,b\right)$上可导,且满足$f\left(a\right)=f\left(b\right)$,则在$\left(a,b\right)$内至少存在一点$\xi$,使得$f^\prime\left(\xi\right)=0$
\item \textbf{拉格朗日中值定理}:若$f\left(x\right)$在$\left[a,b\right]$上连续且在$\left(a,b\right)$上可导,则在$\left(a,b\right)$内至少存在一点$\xi$,使得$f^\prime\left(\xi\right)=\frac{f\left(b\right)-f\left(a\right)}{b-a}$
\item \textbf{柯西中值定理}:若$f\left(x\right)$与$F\left(x\right)$均满足在$\left[a,b\right]$上连续且在$\left(a,b\right)$上可导,且满足对任意$x\in\left(a,b\right)$我们有$F^\prime\left(x\right)\neq0$,则在$\left(a,b\right)$内至少存在一点$\xi$,使得$\frac{f\left(b\right)-f\left(a\right)}{F\left(b\right)-F\left(a\right)}=\frac{f^\prime\left(\xi\right)}{F^\prime\left(\xi\right)}$
\end{enumerate}

\subsection{泰勒公式}
\begin{enumerate}
\item $f\left(x\right)$在$x_0$处的展开式为:
\[f\left(x\right)=f\left(x_0\right)+f^\prime\left(x_0\right)\left(x-x_0\right)+\frac{f^{\prime\prime}\left(x_0\right)}{2!}\left(x-x_0\right)^2+\ldots+\frac{f^{\left(n\right)}\left(x_0\right)}{n!}\left(x-x_0\right)^n+\frac{f^{\left(n+1\right)}\left(\xi\right)}{\left(n+1\right)!}\left(x-x_0\right)^{n+1}\]
其中$\xi$介于$x$与$x_0$之间
\item 常用泰勒展开式:
\begin{enumerate}
\item $e^x=1+x+\dfrac{x^2}{2!}+\cdots+\dfrac{x^n}{n!}+o\left(x^n\right)$
\item $\sin\left(x\right)=x-\dfrac{x^3}{3!}+\dfrac{x^5}{5!}-\cdots+\left(-1\right)^n\dfrac{x^{2n+1}}{\left(2n+1\right)!}+o\left(x^{2n+1}\right)$
\item $\cos\left(x\right)=1-\dfrac{x^2}{2!}+\dfrac{x^4}{4!}-\cdots+\left(-1\right)^n\dfrac{x^{2n}}{\left(2n\right)!}+o\left(x^{2n}\right)$
\item $\ln\left(1+x\right)=x-\dfrac{x^2}{2}+\dfrac{x^3}{3}-\cdots+\left(-1\right)^n\dfrac{x^{n+1}}{n+1}+o\left(x^{n+1}\right)$
\item $\dfrac{1}{1-x}=1+x+x^2+\cdots+x^n+o\left(x^n\right)$
\end{enumerate}
\end{enumerate}

\subsection{曲率}
\begin{enumerate}
\item \textbf{弧微分公式}:$ds=\sqrt{1+\left(y^\prime\right)^2}dx$
\item \textbf{曲率公式}:$K=\dfrac{\left|y^{\prime\prime}\right|}{\left(1+\left(y^\prime\right)^2\right)^\frac{3}{2}}$
\item \textbf{曲率半径}:$\rho=\dfrac{1}{K}$
\end{enumerate}

\subsection{中值定理辅助函数构造表(该表也可用于解决积分问题)}
\begin{enumerate}
\item $\left(\dfrac{1-\xi^2}{\left(1+\xi^2\right)^2}\right)=0\Rightarrow\left[\dfrac{x}{1+x^2}\right]^\prime=0$
\item $f^\prime\left(\xi\right)g\left(\xi\right)+f\left(\xi\right)g^\prime\left(\xi\right)=0\Rightarrow\left[f\left(x\right)g\left(x\right)\right]^\prime=0$
\item $f^\prime\left(\xi\right)g\left(\xi\right)-f\left(\xi\right)g^\prime\left(\xi\right)=0\Rightarrow\left[\dfrac{f\left(x\right)}{g\left(x\right)}\right]^\prime=0$
\item $f\left(\xi\right)g^{\prime\prime}\left(\xi\right)-f^{\prime\prime}\left(\xi\right)g\left(\xi\right)=0\Rightarrow\left[f\left(x\right)g^\prime\left(x\right)-f^\prime\left(x\right)g\left(x\right)\right]^\prime=0$
\item $\left(\xi-a\right)f^\prime\left(\xi\right)+kf\left(\xi\right)=0\Rightarrow\left[\left(x-a\right)^kf\left(x\right)\right]^\prime=0$
\item $\left(a-\xi\right)f^\prime\left(\xi\right)-kf\left(\xi\right)=0\Rightarrow\left[\left(a-x\right)^kf\left(x\right)\right]^\prime=0$
\item $f^\prime\left(\xi\right)g\left(a-\xi\right)-kf\left(\xi\right)g^\prime\left(a-\xi\right)=0\Rightarrow\left[g^k\left(a-x\right)f\left(x\right)\right]^\prime=0$
\item $f^\prime\left(\xi\right)+\lambda f\left(\xi\right)=0\Rightarrow\left[e^{\lambda x}f\left(x\right)\right]^\prime=0$
\item $f^\prime\left(\xi\right)-\lambda f\left(\xi\right)=0\Rightarrow\left[\dfrac{f\left(x\right)}{e^{\lambda x}}\right]^\prime=0$
\item $f^\prime\left(\xi\right)+g^\prime\left(\xi\right)f\left(\xi\right)=0\Rightarrow\left[e^{g\left(x\right)}f\left(x\right)\right]^\prime=0$
\end{enumerate}

\subsection{不等式定理}
\begin{enumerate}
\item 如果
\begin{enumerate}
\item $f\left(x\right)$,$g\left(x\right)$在$\left(a,b\right)$内有n+1阶导数
\item $f^{\left(k\right)}\left(x_0\right)=g^{\left(k\right)}\left(x_0\right)\quad\left(k=0,1,2,\ldots,n,\quad x_0\in\left(a,b\right)\right)$
\item $x>x_0$,且$x\in\left(a,b\right)$时,$f^{\left(n+1\right)}\left(x\right)>g^{\left(n+1\right)}\left(x\right)$(或$f^{\left(n+1\right)}\left(x\right)<g^{\left(n+1\right)}\left(x\right)$)
\end{enumerate}
则当$x>x_0$时,有$f\left(x\right)>g\left(x\right)$(或$f\left(x\right)<g\left(x\right)$)
\end{enumerate}

\subsection{渐近线}
\begin{enumerate}
\item \textbf{水平渐近线}:$\lim\limits_{x\to+\infty}f\left(x\right)=a$,$\lim\limits_{x\to-\infty}f\left(x\right)=b$
\item \textbf{铅直渐近线}:$\lim\limits_{x\to x_0}f\left(x\right)=\infty$
\item \textbf{斜渐近线}:$\left(y=kx+b\right)$:$k=\lim\limits_{x\to+\infty or-\infty}\dfrac{f\left(x\right)}{x}$,$b=\lim\limits_{x\to+\infty or-\infty}\left[f\left(x\right)-kx\right]$
\end{enumerate}


\section{不定积分}
\subsection{积分表}
\begin{enumerate}
\item $\int{\sec^2xdx}=\int\dfrac{dx}{\cos^2x}=\tan x+C$
\item $\int{\csc^2xdx}=\int\dfrac{dx}{\sin^2x}=-\cot x+C$
\item $\int{\sec x\tan xdx}=\sec x+C$
\item $\int{\csc x\cot xdx}=-\csc x+C$
\item $\int{\tan xdx}=-\ln\left|\cos x\right|+C$
\item $\int{\cot xdx}=\ln\left|\sin x\right|+C$
\item $\int{\sec xdx}=\ln\left|\sec x+\tan x\right|+C$
\item $\int{\csc xdx}=\ln\left|\csc x-\cot x\right|+C$
\item $\int{\dfrac{dx}{x^2+a^2}}=\dfrac{1}{a}\arctan\dfrac{1}{a}+C$
\item $\int{\dfrac{dx}{x^2-a^2}}=\dfrac{1}{2a}\ln\left|\dfrac{x-a}{x+a}\right|+C$
\item $\int{\dfrac{dx}{\sqrt{x^2+a^2}}}=\ln\left(x+\sqrt{x^2+a^2}\right)+C$
\item $\int{\dfrac{dx}{\sqrt{x^2-a^2}}}=\ln\left|x+\sqrt{x^2-a^2}\right|+C$
\item $\int{\dfrac{dx}{\sqrt{a^2-x^2}}}=\arcsin\dfrac{x}{a}+C$
\end{enumerate}

\subsection{几个定理}
\begin{enumerate}
\item 若连续函数$f\left(x\right)$在区间$I$上连续,则$f\left(x\right)$在$I$上有原函数
\item 若连续函数$f\left(x\right)$在区间$I$上有原函数,且$x_0\in I$是$f\left(x\right)$的不连续点,则$x_0$必为$f\left(x\right)$的第二类间断点,换言之,有第一类间断点的函数不存在原函数
\end{enumerate}


\section{定积分}
\subsection{积分中值定理}
\begin{enumerate}
\item \textbf{定积分中值定理}:如果函数$f\left(x\right)$在积分区间$\left[a,b\right]$上连续,则在区间$\left[a,b\right]$上至少存在一个点$\xi$,使得下式成立:
\[\int_{a}^{b}f\left(x\right)dx=f\left(\xi\right)\left(b-a\right)\]
\item \textbf{推广的积分中值定理}:
\begin{enumerate}
\item \textbf{第一定理}:如果函数$f\left(x\right)$,$g\left(x\right)$在闭区间$\left[a,b\right]$上可积,$f\left(x\right)$连续且$g\left(x\right)$在$\left[a,b\right]$上不变号,则在积分区间$\left[a,b\right]$上至少存在一个点$\xi$,使下式成立:
\[\int_{a}^{b}f\left(x\right)g\left(x\right)dx=f\left(\xi\right)\int_{a}^{b}g\left(x\right)dx\]
\item \textbf{第二定理}:
\begin{enumerate}
\item 如果函数$f\left(x\right)$,$g\left(x\right)$在闭区间$\left[a,b\right]$上可积,且$f\left(x\right)$为单调函数,则在积分区间$\left[a,b\right]$上至少存在一个点$\xi$,使下式成立:
\[\int_{a}^{b}f\left(x\right)g\left(x\right)dx=f\left(a\right)\int_{a}^{\xi}g\left(x\right)dx+f\left(b\right)\int_{\xi}^{b}g\left(x\right)dx\]
\item 如果函数$f\left(x\right)$,$g\left(x\right)$在闭区间$\left[a,b\right]$上可积,且$f\left(x\right)\geq0$并为单调递减函数,则在积分区间$\left[a,b\right]$上至少存在一个点$\xi$,使下式成立:
\[\int_{a}^{b}f\left(x\right)g\left(x\right)dx=f\left(a\right)\int_{a}^{\xi}g\left(x\right)dx\]
\item 如果函数$f\left(x\right)$,$g\left(x\right)$在闭区间$\left[a,b\right]$上可积,且$f\left(x\right)\geq0$并为单调递增函数,则在积分区间$\left[a,b\right]$上至少存在一个点$\xi$,使下式成立:
\[\int_{a}^{b}f\left(x\right)g\left(x\right)dx=f\left(b\right)\int_{\xi}^{b}g\left(x\right)dx\]
\end{enumerate}
\end{enumerate}
\end{enumerate}

\subsection{反常积分收敛}
\begin{enumerate}
\item \textbf{反常积分收敛定义}
\begin{enumerate}
\item \textbf{无穷限反常积分}:设函数$f\left(x\right)$在区间$\left[a,+\infty\right)$上连续,任取$t>a$,若极限$\lim\limits_{t\to+\infty}\int_{a}^{t}f\left(x\right)dx$存在,则称反常积分$\int_{a}^{+\infty}f\left(x\right)dx$收敛,且有
\[\int_{a}^{+\infty}f\left(x\right)dx=\lim\limits_{t\to+\infty}\int_{a}^{t}f\left(x\right)dx\]
同理,当函数$f\left(x\right)$在区间$\left(-\infty,b\right]$和$\left(-\infty,+\infty\right)$上连续时我们分别有
\[\int_{-\infty}^{b}f\left(x\right)dx=\lim\limits_{t\to-\infty}\int_{t}^{b}f\left(x\right)dx\]
\[\int_{-\infty}^{+\infty}f\left(x\right)dx=\lim\limits_{t_1\to-\infty}\int_{t_1}^{k}f\left(x\right)dx+\lim\limits_{t_2\to+\infty}\int_{k}^{t_2}f\left(x\right)dx\quad(k\text{为任意常数})\]
\item \textbf{无界函数反常积分}:设函数$f\left(x\right)$在区间$\left(a,b\right]$上连续,$a$为函数$f\left(x\right)$的瑕点(即无穷间断点),任取$t\in\left(a,b\right]$,若极限$\lim\limits_{t\to a^+}\int_{t}^{b}f\left(x\right)dx$存在,则称反常积分$\int_{a}^{b}f\left(x\right)dx$收敛,且有
\[\int_{a}^{b}f\left(x\right)dx=\lim\limits_{t\to a^+}\int_{t}^{b}f\left(x\right)dx\]
同理,当函数$f\left(x\right)$在区间$\left[a,b\right)$上连续且瑕点为$b$以及$\left[a,c\right)\cup\left(c,b\right]$上连续且瑕点为$c$时我们分别有
\[\int_{a}^{b}f\left(x\right)dx=\lim\limits_{t\to b^-}\int_{a}^{t}f\left(x\right)dx\]
\[\int_{a}^{b}f\left(x\right)dx=\lim\limits_{t_1\to c^-}\int_{a}^{t_1}f\left(x\right)dx+\lim\limits_{t_2\to c^+}\int_{t_2}^{b}f\left(x\right)dx\]
\end{enumerate}
\item \textbf{绝对收敛}:对于反常积分(无穷限反常积分或者无界函数反常积分)$\int f\left(x\right)dx$,若反常积分$\int\left|f\left(x\right)\right|dx$收敛,则称反常积分$\int f\left(x\right)dx$绝对收敛。反常积分绝对收敛则原反常积分一定收敛
\item \textbf{条件收敛}:对于反常积分(无穷限反常积分或者无界函数反常积分)$\int f\left(x\right)dx$,若反常积分$\int f\left(x\right)dx$收敛但反常积分$\int\left|f\left(x\right)\right|dx$发散,则称反常积分$\int f\left(x\right)dx$条件收敛
\end{enumerate}

\subsection{无穷限反常积分审敛法}
\begin{enumerate}
\item \textbf{定理}:对于无穷限反常积分$\int_{a}^{+\infty}f\left(x\right)dx$,假设$f\left(x\right)$在区间$\left[a,+\infty\right)\quad\left(a>0\right)$上连续,且$f\left(x\right)\geq0$,若我们有$\lim\limits_{x\to+\infty}\frac{f\left(x\right)}{g\left(x\right)}=l$,则:
\begin{enumerate}
\item 当$l=0$时,若$\int_{a}^{+\infty}g\left(x\right)dx$收敛,则$\int_{a}^{+\infty}f\left(x\right)dx$收敛
\item 当$l=+\infty$时,若$\int_{a}^{+\infty}g\left(x\right)dx$发散,则$\int_{a}^{+\infty}f\left(x\right)dx$发散
\item 当$0<l<+\infty$时,$\int_{a}^{+\infty}g\left(x\right)dx$与$\int_{a}^{+\infty}f\left(x\right)dx$敛散性相同
\end{enumerate}
\item \textbf{比较审敛法}:\\
设函数$f\left(x\right)$在区间$\left[a,+\infty\right)\quad\left(a>0\right)$上连续,且$f\left(x\right)\geq0$,
\begin{enumerate}
\item 若存在常数$M>0$及$p>1$,使得$f\left(x\right)\le\frac{M}{x^p}\quad\left(a\le x<+\infty\right)$,则反常积分$\int_{a}^{+\infty}f\left(x\right)dx$收敛;
\item 若存在常数$N>0$,使得$f\left(x\right)\geq\frac{N}{x}\quad\left(a\le x<+\infty\right)$,则反常积分$\int_{a}^{+\infty}f\left(x\right)dx$发散
\end{enumerate}
\item \textbf{极限审敛法}:\\
设函数$f\left(x\right)$在区间$\left[a,+\infty\right)\quad\left(a>0\right)$上连续,且$f\left(x\right)\geq0$,
\begin{enumerate}
\item 若存在常数$p>1$,使得$\lim\limits_{x\to+\infty}x^pf\left(x\right)=c<+\infty$,则反常积分$\int_{a}^{+\infty}f\left(x\right)dx$收敛;
\item 若$\lim\limits_{x\to+\infty}xf\left(x\right)=d>0$或$\lim\limits_{x\to+\infty}{xf\left(x\right)}=+\infty$,则反常积分$\int_{a}^{+\infty}f\left(x\right)dx$发散
\end{enumerate}
\end{enumerate}

\subsection{无界函数反常积分审敛法}
\begin{enumerate}
\item \textbf{定理}:对于无界函数反常积分$\int_{a}^{b}f\left(x\right)dx$,假设$f\left(x\right)$在区间$\left(a,b\right]$上连续,且$f\left(x\right)\geq0$,$x=a$为$f\left(x\right)$的瑕点,若我们有$\lim\limits_{x\to a^{+}}\frac{f\left(x\right)}{g\left(x\right)}=l$,则:
\begin{enumerate}
\item 当$l=0$时,若$\int_{a}^{+\infty}g\left(x\right)dx$收敛,则$\int_{a}^{+\infty}f\left(x\right)dx$收敛
\item 当$l=+\infty$时,若$\int_{a}^{+\infty}g\left(x\right)dx$发散,则$\int_{a}^{+\infty}f\left(x\right)dx$发散
\item 当$0<l<+\infty$时,$\int_{a}^{+\infty}g\left(x\right)dx$与$\int_{a}^{+\infty}f\left(x\right)dx$敛散性相同
\end{enumerate}
\item \textbf{比较审敛法}:\\
设函数$f\left(x\right)$在区间$\left(a,b\right]$上连续,且$f\left(x\right)\geq0$,$x=a$为$f\left(x\right)$的瑕点,
\begin{enumerate}
\item 若存在常数$M>0$及$0<q<1$,使得$f\left(x\right)\le\frac{M}{\left(x-a\right)^q}\quad\left(a<x\le b\right)$,则反常积分$\int_{a}^{b}f\left(x\right)dx$收敛;
\item 若存在常数$N>0$及$q\geq1$,使得$f\left(x\right)\geq\frac{N}{\left(x-a\right)^q}\quad\left(a<x\le b\right)$,则反常积分$\int_{a}^{b}f\left(x\right)dx$发散
\end{enumerate}
\item \textbf{极限审敛法}:\\
设函数$f\left(x\right)$在区间$\left(a,b\right]$上连续,且$f\left(x\right)\geq0$,$x=a$为$f\left(x\right)$的瑕点,
\begin{enumerate}
\item 若存在常数$0<q<1$,使得$\lim\limits_{x\to a^{+}}\left(x-a\right)^qf\left(x\right)=c<+\infty$,则反常积分$\int_{a}^{b}f\left(x\right)dx$收敛;
\item 若$\lim\limits_{x\to a^{+}}\left(x-a\right)f\left(x\right)=d>0$或$\lim\limits_{x\rightarrow a^{+}}\left(x-a\right)f\left(x\right)=+\infty$,则反常积分$\int_{a}^{b}f\left(x\right)dx$发散
\end{enumerate}
\end{enumerate}

\subsection{$\Gamma$函数}
\begin{enumerate}
\item $\Gamma$函数的定义为:
\[\Gamma\left(s\right)=\int_{0}^{+\infty}e^{-x}x^{s-1}dx\quad\left(s>0\right)\]
\item 计算性质:
\begin{enumerate}
\item $\Gamma\left(1\right)=1,\ \Gamma\left(\frac{1}{2}\right)=\sqrt{\pi}$
\item $\Gamma\left(s+1\right)=s\Gamma\left(s\right)$(即我们有,对任意正整数$n$,$\Gamma\left(n+1\right)=n!$)
\item $\Gamma\left(s\right)\Gamma\left(1-s\right)=\frac{\pi}{\sin\pi s}$
\item $\lim\limits_{s\to 0^{+}}\Gamma\left(s\right)=+\infty$
\end{enumerate}
\end{enumerate}

\subsection{三角函数一些特殊积分公式}
\begin{enumerate}
\item $\int_0^{\frac{\pi}{2}}\sin^nxdx=\int_0^{\frac{\pi}{2}}\cos^nxdx=
\left\{\begin{aligned}
&\frac{\pi}{2}\cdot\frac{\left(n-1\right)!!}{n!!},\quad n\text{为正偶数}\\
&\frac{\left(n-1\right)!!}{n!!},\quad n\text{为大于1的奇数}
\end{aligned}\right.$
\item $\int_0^{\frac{\pi}{2}}f\left(\sin x\right)dx=\int_0^{\frac{\pi}{2}}f\left(\cos x\right)dx$
\item $\int_0^{\frac{\pi}{2}}f\left(\sin x,\cos x\right)dx=\int_0^{\frac{\pi}{2}}f\left(\cos x,\sin x\right)dx$
\item $\int_0^{\pi}f\left(\sin x\right)dx=2\int_0^{\frac{\pi}{2}}f\left(\sin x\right)dx$,特别有$\int_0^{\pi}\sin^nxdx=2\int_0^{\frac{\pi}{2}}\sin^nxdx$
\item $\int_0^{\pi}xf\left(\sin x\right)dx=\frac{\pi}{2}\int_0^{\pi}f\left(\sin x\right)dx$
\item $\int_0^{\pi}\cos^nxdx=
\left\{\begin{aligned}
&2\int_0^{\frac{\pi}{2}}\cos^nxdx,\quad n\text{为偶数}\\
&0,\quad n\text{为奇数}
\end{aligned}\right.$
\item $\int_0^{2\pi}xf\left(\cos x\right)dx=2\pi\int_0^{\pi}f\left(\cos x\right)dx$
\end{enumerate}

\subsection{变限积分求导公式}
\begin{enumerate}
\item $\frac{d}{dx}\int_{\psi\left(x\right)}^{\varphi\left(x\right)}f\left(t\right)dt=f\left[\varphi\left(x\right)\right]\varphi^\prime\left(x\right)-f\left[\psi\left(x\right)\right]\psi^\prime\left(x\right)$
\end{enumerate}

\subsection{定积分偏导公式}
\begin{enumerate}
\item $\frac{d}{dx}\int_{a}^{b}f\left(x,t\right)dt=\int_{a}^{b}\frac{\partial f\left(x,t\right)}{\partial x}dt$
\end{enumerate}

\subsection{柯西-施瓦兹不等式}
\begin{enumerate}
\item 设$f\left(x\right)$与$g\left(x\right)$在区间$\left[a,b\right]$上均连续,则我们有:
\[\left[\int_{a}^{b}f\left(x\right)g\left(x\right)dx\right]^2\le\left(\int_{a}^{b}f^2\left(x\right)dx\right)\left(\int_{a}^{b}g^2\left(x\right)dx\right)\]
当且仅当$f\left(x\right)=Cg\left(x\right)$时等号成立($C$为常数)
\end{enumerate}


\section{微分方程}
\subsection{初等积分方法}
\begin{enumerate}
\item 一阶线性微分方程:$\frac{dy}{dx}=a\left(x\right)y+f\left(x\right)$\\
解:常数变易法,先求解$\frac{dy}{dx}=a\left(x\right)y$,可得到通解为$y=Ce^{\int{a\left(x\right)dx}}$,再假设原方程的解为$y=C\left(x\right)e^{\int{a\left(x\right)dx}}$,带入即可解得$C\left(x\right)$,此时方程得解
\item 伯努利方程:$\frac{dy}{dx}=a\left(x\right)y+f\left(x\right)y^{\alpha}$\\
解:方程两段同时除以$y^{\alpha}$,此时方程变为$y^{-\alpha}\frac{dy}{dx}=a\left(x\right)y^{1-\alpha}+f\left(x\right)$,设$z=y^{1-\alpha}$,此时我们有$\frac{dz}{dx}=\left(1-\alpha\right)y^{-\alpha}\frac{dy}{dx}$,于是方程又可变为$\frac{dz}{dx}=\left(1-\alpha\right)a\left(x\right)z+\left(1-\alpha\right)f\left(x\right)$,此时方程便化为了(1)中的一阶线性微分方程的问题了
\item 齐次方程:$\frac{dy}{dx}=f\left(x,y\right)$,其中$f\left(x,y\right)$是$x$,$y$的齐次函数\\
解:设$u=\frac{y}{x}$,将$u$带入原式做变量代换后原式可化为变量分离形式方程
\item 线性分式形式微分方程:$\frac{dy}{dx}=f\left(\frac{a_1x+b_1y+c_1}{a_2x+b_2y+c_2}\right)$\\
解:若$\begin{vmatrix}a_1&b_1\\a_2&b_2\end{vmatrix}\ne0$,则可以先求解方程组$\left\{\begin{aligned}&a_1x+b_1y+c_1=0\\&a_2x+b_2y+c_2=0\end{aligned}\right.$,假设解为$x_0$和$y_0$,则此时对原方程做平移变换$\xi=x-x_0$和$\eta=y-y_0$可以将原式化为齐次方程$\frac{d\eta}{d\xi}=f\left(\frac{a_1\xi+b_1\eta}{a_2\xi+b_2\eta}\right)$;\\
若$\begin{vmatrix}a_1&b_1\\a_2&b_2\end{vmatrix}=0$,此时$a_1=a_2=0$或$b_1=b_2=0$,方程已经为分离变量形式,可以直接求解;假如$\frac{a_2}{a_1}=\frac{b_2}{b_1}=\lambda\ne0$或$a_1=b_1=0$,可以引入新变量$u=a_2x+b_2y$再代入原式将其化为分离变量形式
\end{enumerate}

\subsection{恰当方程}
\begin{enumerate}
\item 对于一阶微分方程$M\left(x,y\right)dx+N\left(x,y\right)dy=0$,若方程满足$\frac{\partial M}{\partial y}=\frac{\partial N}{\partial x}$,则该一阶微分方程即为恰当方程\\
对于某些不是恰当方程的一阶微分方程$M\left(x,y\right)dx+N\left(x,y\right)dy=0$,有时可以方程两端乘上一个积分因子$\mu\left(x,y\right)$使其成为恰当方程,即乘上$\mu\left(x,y\right)$后的微分方程$\mu\left(x,y\right)M\left(x,y\right)dx+\mu\left(x,y\right)N\left(x,y\right)dy=0$可以变为恰当方程。积分因子$\mu\left(x,y\right)$有时可以目测得到,有时也可以通过下述方法得到:\\
记$E\left(x,y\right)=\frac{\partial M}{\partial y}-\frac{\partial N}{\partial x}$并称其为恰当判别式,若$\frac{E}{N}$仅为变量$x$的函数而与变量$y$无关时,原微分方程存在仅与变量$x$有关的积分因子$\mu\left(x\right)=e^{\int{\frac{E}{N}dx}}$;同理,若$\frac{E}{M}$仅为变量$y$的函数而与变量$x$无关时,原微分方程存在仅与变量$y$有关的积分因子$\mu\left(y\right)=e^{-\int{\frac{E}{M}dy}}$
\end{enumerate}

\subsection{隐式方程(注意此时方程可能为一阶高次微分方程)}
\begin{enumerate}
\item 对于一般的隐式方程$F\left(x,y,\frac{dy}{dx}\right)=0$,我们设$p=\frac{dy}{dx}$并将其看作代数方程$F\left(x,y,p\right)=0$的独立变量,现在我们再将$F\left(x,y,p\right)=0$表示为参数形式$\left\{\begin{aligned}&x=a\left(s,t\right)\\&y=b\left(s,t\right)\\&p=c\left(s,t\right)\end{aligned}\right.$,再分别对参数方程中的$x=a\left(s,t\right)$与$y=b\left(s,t\right)$求全微分可得$\left\{\begin{aligned}&dx=\frac{\partial a}{\partial s}ds+\frac{\partial a}{\partial t}dt\\&dy=\frac{\partial b}{\partial s}ds+\frac{\partial b}{\partial t}dt\end{aligned}\right.$,将这两个全微分带入$dy=\frac{dy}{dx}dx=c\left(s,t\right)dx$并化简可以得到一阶线性微分方程$\left(\frac{\partial b}{\partial s}-c\left(s,t\right)\frac{\partial a}{\partial s}\right)ds+\left(\frac{\partial b}{\partial t}-c\left(s,t\right)\frac{\partial a}{\partial t}\right)dt=0$,求解该方程后即可得到$s$和$t$的函数,带入原参数方程即可解得原隐式方程的参数形式解\\
几种常见的易求解的隐式方程有:
\begin{enumerate}
\item 可以解出$y$的方程$y=f\left(x,\frac{dy}{dx}\right)$
\item 可以解出$x$的方程$x=f\left(y,\frac{dy}{dx}\right)$
\item 不含$y$的方程$f\left(x,\frac{dy}{dx}\right)=0$
\item 不含$x$的方程$f\left(y,\frac{dy}{dx}\right)=0$
\end{enumerate}
\end{enumerate}

\subsection{高阶微分方程(注意与隐式方程的高次方程区别,一个为高阶一个为一阶高次)}
\begin{enumerate}
\item 高阶方程没有一般解法,基本思路为降阶,以下为几种可以降阶求解的高阶微分方程
\begin{enumerate}
\item 不显含$y$的方程$F\left(x,\frac{d^ky}{dx^k},\cdots,\frac{d^ny}{dx^n}\right)=0$,令$p=\frac{d^ky}{dx^k}$,带入原方程即可降阶,求出$p$后再反求出$y$
\item 不显含$x$的方程$F\left(y,\frac{d^ky}{dx^k},\cdots,\frac{d^ny}{dx^n}\right)=0$,令$p=\frac{dy}{dx}$,带入原方程即可降阶,求出$p$后再反求出$y$
\item 关于变量$y,\frac{dy}{dx},\cdots,\frac{d^ny}{dx^n}$(注意不包括$x$)的齐次方程$F\left(x,y,\frac{d^ky}{dx^k},\cdots,\frac{d^ny}{dx^n}\right)=0$,令$p=\frac{1}{y}\frac{dy}{dx}$,带入原方程即可降阶,求出$p$后再反求出$y$
\item 全微分方程,可以查看方程是否为某个函数的全微分,或者乘上某个积分因子后为全微分
\end{enumerate}
\end{enumerate}

\subsection{常系数线性方程$\frac{d^ny}{dx^n}+a_1\frac{d^{n-1}y}{dx^{n-1}}+\cdots+a_{n-1}\frac{dy}{dx}+a_ny=f\left(x\right)$}
\begin{enumerate}
\item 引入微分算子记号$D=\frac{d}{dx}$,即我们有$D^n=\frac{d^n}{dt^n}$,首先明确思路,我们知道非齐次常系数线性方程$\frac{d^ny}{dx^n}+a_1\frac{d^{n-1}y}{dx^{n-1}}+\cdots+a_{n-1}\frac{dy}{dx}+a_ny=f\left(x\right)$的通解可以由齐次常系数线性方程$\frac{d^ny}{dx^n}+a_1\frac{d^{n-1}y}{dx^{n-1}}+\cdots+a_{n-1}\frac{dy}{dx}+a_ny=0$的通解和非齐次常系数线性方程本身的一个特解相加得到,故我们分两步进行求解
\begin{enumerate}
\item 求齐次常系数线性方程$\frac{d^ny}{dx^n}+a_1\frac{d^{n-1}y}{dx^{n-1}}+\cdots+a_{n-1}\frac{dy}{dx}+a_ny=0$的通解\\
将齐次常系数线性方程写为微分算子形式可得:
\[\left(D^n+a_1D^{n-1}+\cdots+a_{n-1}D+a_n\right)y=0\]
设$P\left(\lambda\right)={\lambda}^n+a_1{\lambda}^{n-1}+\cdots+a_{n-1}\lambda+a_n$,我们称该$P\left(\lambda\right)$为原方程的特征多项式,显然此时原齐次常系数微分方程也可简记为$P\left(D\right)y=0$\\
现在来通过特征多项式来求齐次方程通解,假设求解特征方程$P\left(\lambda\right)=0$得到了$r$个互异的实特征根$\lambda_1,\lambda_2,\cdots,\lambda_r$以及$l$对互异的复特征根$\alpha_1\pm i\beta_1,\alpha_2\pm i\beta_2,\cdots,\alpha_l\pm i\beta_l$,重数分别为$n_1,n_2,\cdots,n_r$和$m_1,m_2,\cdots,m_l$,则齐次常系数线性微分方程的基本解组为:
\[\begin{matrix}
e^{\lambda_1x},&xe^{\lambda_1x},&\cdots,&x^{n_1-1}e^{\lambda_1x}\\
\vdots&\ddots&\ddots&\ddots\\
e^{\lambda_rx},&xe^{\lambda_rx},&\cdots,&x^{n_r-1}e^{\lambda_rx}\\
e^{\alpha_1x}\cos\beta_1x,&xe^{\alpha_1x}\cos\beta_1x,&\cdots,&x^{m_1-1}e^{\alpha_1x}\cos\beta_1x\\
\vdots&\ddots&\ddots&\ddots\\
e^{\alpha_lx}\cos\beta_lx,&xe^{\alpha_lx}\cos\beta_lx,&\cdots,&x^{m_l-1}e^{\alpha_lx}\cos\beta_lx\\
e^{\alpha_1x}\sin\beta_1x,&xe^{\alpha_1x}\sin\beta_1x,&\cdots,&x^{m_1-1}e^{\alpha_1x}\sin\beta_1x\\
\vdots&\ddots&\ddots&\ddots\\
e^{\alpha_lx}\sin\beta_lx,&xe^{\alpha_lx}\sin\beta_lx,&\cdots,&x^{m_l-1}e^{\alpha_lx}\sin\beta_lx\\
\end{matrix}\]
\item 求非齐次常系数线性方程$\frac{d^ny}{dx^n}+a_1\frac{d^{n-1}y}{dx^{n-1}}+\cdots+a_{n-1}\frac{dy}{dx}+a_ny=f\left(x\right)$的一个特解\\
由$P\left(D\right)y=f\left(x\right)$可以得到$y=\frac{1}{P\left(D\right)}f\left(x\right)$,而算子$\frac{1}{P\left(D\right)}$具有如下性质:\\
性质1:$\frac{1}{D^n}f\left(x\right)=\int\cdots\int f\left(t\right)\left(dt\right)^n$,即$n$累次积分\\
性质2:$\frac{1}{P\left(D\right)}$的作用是线性的,即:
\[\frac{1}{P\left(D\right)}\left\{\alpha f_1\left(x\right)+\beta f_2\left(x\right)\right\}=\alpha\frac{1}{P\left(D\right)}f_1\left(x\right)+\beta\frac{1}{P\left(D\right)}f_2\left(x\right)\]
性质3:若$P\left(D\right)=P_1\left(D\right)P_2\left(D\right)$,则:
\[\frac{1}{P\left(D\right)}=\frac{1}{P_1\left(D\right)}\cdot\frac{1}{P_2\left(D\right)}=\frac{1}{P_2\left(D\right)}\cdot\frac{1}{P_1\left(D\right)}\]
性质4:对于一个$k$次多项式$f_k\left(x\right)$,若函数$\frac{1}{P\left(x\right)}$在$x=0$处解析且可以展开成$\frac{1}{P\left(x\right)}=Q_k\left(x\right)+H_k\left(x\right)$,其中$Q_k\left(x\right)$是$k$次多项式而$H_k\left(x\right)$为$k+1$次以上的所有高次项,则我们有:
\[\frac{1}{P\left(D\right)}f_k\left(x\right)=Q_k\left(D\right)f_k\left(x\right)\]
性质5:如果$P\left(\lambda\right)\ne0$,那么$\frac{1}{P\left(D\right)}e^{\lambda x}=\frac{1}{P\left(\lambda\right)}e^{\lambda x}$\\
性质6:$\frac{1}{P\left(D\right)}e^{\lambda x}v\left(x\right)=e^{\lambda x}\frac{1}{P\left(D+\lambda\right)}v\left(x\right)$\\
通过上述性质,通常可以求解出一个特解
\end{enumerate}
\end{enumerate}


\section{解析几何}
\subsection{方向角、方向余弦与投影}
\begin{enumerate}
\item 非零空间向量$\vec{r}$与三条坐标轴的夹角$\alpha,\beta,\gamma$称为向量$\vec{r}$的方向角,而与$\vec{r}$同方向的单位向量$\vec{e_r}=\left(\cos\alpha,\cos\beta,\cos\gamma\right)$的三个分量$\cos\alpha$、$\cos\beta$、$\cos\gamma$则被称为向量$\vec{r}$的方向余弦,向量$\vec{r}$在向量$\vec{a}$上的投影为:$\operatorname{Prj}_{\vec{a}}\vec{r}=\left|\vec{r}\right|\cos\theta$,其中$\theta$为向量$\vec{r}$与向量$\vec{a}$的夹角
\end{enumerate}

\subsection{向量积的运算规律}
\begin{enumerate}
\item 向量积大小为$\left|\vec{a}\times\vec{b}\right|=\left|\vec{a}\right|\left|\vec{b}\right|\sin\theta$,方向满足右手系关系
\item 设$\vec{a}=a_x\vec{i}+a_y\vec{j}+a_z\vec{k}$,$\vec{b}=b_x\vec{i}+b_y\vec{j}+b_z\vec{k}$,则:
\[\vec{a}\times\vec{b}=\begin{vmatrix}\vec{i}&\vec{j}&\vec{k}\\a_x&a_y&a_z\\b_x&b_y&b_z\end{vmatrix}=\begin{vmatrix}a_y&a_z\\b_y&b_z\end{vmatrix}\vec{i}+\begin{vmatrix}a_z&a_x\\b_z&b_x\end{vmatrix}\vec{j}+\begin{vmatrix}a_x&a_y\\b_x&b_y\end{vmatrix}\vec{k}\]
\item $\vec{a}\times\left(\vec{b}\times\vec{c}\right)=\left(\vec{a}\cdot\vec{c}\right)\vec{b}-\left(\vec{a}\cdot\vec{b}\right)\vec{c}$
\item 混合积公式:$\left[\vec{a}\quad\vec{b}\quad\vec{c}\right]=\vec{a}\cdot\left(\vec{b}\times\vec{c}\right)$\\
若设$\vec{a}=a_x\vec{i}+a_y\vec{j}+a_z\vec{k}$,$\vec{b}=b_x\vec{i}+b_y\vec{j}+b_z\vec{k}$,$\vec{c}=c_x\vec{i}+c_y\vec{j}+c_z\vec{k}$,则混合积公式又可写为:
\[\vec{a}\cdot\left(\vec{b}\times\vec{c}\right)=\begin{vmatrix}a_x&a_y&a_z\\b_x&b_y&b_z\\c_x&c_y&c_z\end{vmatrix}\]
\end{enumerate}

\subsection{平面方程}
\begin{enumerate}
\item 平面的一般方程为$Ax+By+Cz+D=0$,从方程可以直接得到它的一个法向量$\vec{n}=\left(A,B,C\right)$。以下为几种平面的特殊表示方法
\begin{enumerate}
\item 点法式方程:设平面的一个法向量为$\vec{n}=\left(A,B,C\right)$,$M_0=\left(x_0,y_0,z_0\right)$为平面上任意一点,则平面方程可以表示为
\[A\left(x-x_0\right)+B\left(y-y_0\right)+C\left(z-z_0\right)=0\]
\item 截距式方程:设平面在$x$、$y$、$z$轴上的截距分别为$a$、$b$、$c$,则平面方程可以表示为
\[\frac{x}{a}+\frac{y}{b}+\frac{z}{c}=1\]
\end{enumerate}
\end{enumerate}

\subsection{直线方程}
\begin{enumerate}
\item 由于直线可以看作两平面的交点集合,即用两个平面相交可以表示一条直线,故直线的一般方程为
\[\left\{\begin{aligned}&A_1x+B_1y+C_1z+D_1=0\\&A_2x+B_2y+C_2z+D_2=0\end{aligned}\right.\]
以下为几种直线的特殊表示方法
\begin{enumerate}
\item 点向式方程(又称对称式方程):设直线的一个方向向量为$\vec{s}=\left(m,n,p\right)$,$M_0=\left(x_0,y_0,z_0\right)$为直线上任意一点,则直线方程可以表示为
\[\frac{x-x_0}{m}=\frac{y-y_0}{n}=\frac{z-z_0}{p}\]
\item 参数方程:设$\frac{x-x_0}{m}=\frac{y-y_0}{n}=\frac{z-z_0}{p}=t$,由于$t$的大小会随$\left(x,y,z\right)$的取值不同而变化,故也可以借此写出直线的参数方程
\[\left\{\begin{aligned}&x=x_0+mt\\&y=y_0+nt\\&z=z_0+pt\end{aligned}\right.\]
\end{enumerate}
\end{enumerate}

\subsection{平面束}
\begin{enumerate}
\item 由于直线方程可以用两个平面相交的交线来表示,故通过直线的一般方程也可以反过来表示出所有过这条直线的平面,即
\[\lambda\left(A_1x+B_1y+C_1z+D_1\right)+\mu\left(A_2x+B_2y+C_2z+D_2\right)=0\]
上述方程即为通过该直线的平面束方程,当$\lambda:\mu$的比值取值不同时表示不同的过该直线的平面,需要注意的是,书上一般写的是单参数,但单参数有个问题是无法表示全部过该直线的平面,例如单参数平面束$\left(A_1x+B_1y+C_1z+D_1\right)+\lambda\left(A_2x+B_2y+C_2z+D_2\right)=0$就无法表示平面$A_2x+B_2y+C_2z+D_2=0$
\end{enumerate}

\subsection{常见二次曲面方程}
\begin{enumerate}
\item 椭圆锥面:$\dfrac{x^2}{a^2}+\dfrac{y^2}{b^2}=z^2$
\item 椭球面:$\dfrac{x^2}{a^2}+\dfrac{y^2}{b^2}+\dfrac{z^2}{c^2}=1$
\item 单叶双曲面:$\dfrac{x^2}{a^2}+\dfrac{y^2}{b^2}-\dfrac{z^2}{c^2}=1$
\item 双叶双曲面:$\dfrac{x^2}{a^2}-\dfrac{y^2}{b^2}-\dfrac{z^2}{c^2}=1$
\item 椭圆抛物面:$\dfrac{x^2}{a^2}+\dfrac{y^2}{b^2}=z$
\item 双曲抛物面:$\dfrac{x^2}{a^2}-\dfrac{y^2}{b^2}=z$
\end{enumerate}

\subsection{空间曲线在坐标面上的投影曲线求法}
\begin{enumerate}
\item 空间曲线$\Gamma$的一般方程为
\[\left\{\begin{aligned}&F\left(x,y,z\right)=0\\&G\left(x,y,z\right)=0\end{aligned}\right.\]
假设投影到$xOy$平面上,上面两式消去$z$后可以得到方程$H\left(x,y\right)=0$,此时曲线在$xOy$平面上的投影即为:
\[\left\{\begin{aligned}&H\left(x,y\right)=0\\&z=0\end{aligned}\right.\]
另外两个面同理
\end{enumerate}

\subsection{点到平面距离}
\begin{enumerate}
\item 设平面外的一点为$M_0=\left(x_0,y_0,z_0\right)$,平面方程为$Ax+By+Cz+D=0$,则点$M_0$到平面的距离公式为:
\[d=\frac{\left|Ax_0+By_0+Cz_0+D\right|}{\sqrt{A^2+B^2+C^2}}\]
至于其他距离求法,点到直线距离可以转化为向量求解,直线或者平面到平面的距离可以转化为直线或者平面上的任意取一点到平面的距离,即转换为点到平面距离的求解
\end{enumerate}


\section{多元函数微分学}
\subsection{几个定理}
\begin{enumerate}
\item 定理(充分条件):如果函数$z=f\left(x,y\right)$的两个二阶混合偏导数$\frac{\partial^2z}{\partial y\partial x}$与$\frac{\partial^2z}{\partial x\partial y}$在区域$D$内连续,则在该区域内这两个二阶混合偏导数必然相等,即二阶偏导数在连续的情况下与求导次序无关
\item 定理(充分条件):如果函数$z=f\left(x,y\right)$的偏导数$\frac{\partial z}{\partial x}$与$\frac{\partial z}{\partial y}$在点$\left(x,y\right)$连续,则在该点处可微分
\item 可微与偏导数的关系:对于二元函数$z=f\left(x,y\right)$,我们有
\begin{enumerate}
\item 可微$\Rightarrow$函数连续,偏导数存在,但偏导数不一定连续
\item 偏导数连续$\Rightarrow$函数连续且可微
\end{enumerate}
\item 由全微分的定义出发,若函数$z=f\left(x,y\right)$的全增量$\Delta z=f\left(x+\Delta x,y+\Delta y\right)-f\left(x,y\right)$可以表示为$\Delta z=A\Delta x+B\Delta y+o\left(\rho\right)$,其中$\rho=\sqrt{\left(\Delta x\right)^2+\left(\Delta y\right)^2}$,则称函数$z=f\left(x,y\right)$可微分。由此出发可以得到,对于一个函数$z=f\left(x,y\right)$,若极限
\[\lim\limits_{\rho\to0}\frac{\Delta z-\left[f_x^{\prime}\left(x_0,y_0\right)\Delta x+f_y^{\prime}\left(x_0,y_0\right)\Delta y\right]}{\rho}=0\]
成立,则函数在点$\left(x_0,y_0\right)$处可微,否则不可微
\end{enumerate}

\subsection{隐函数求导}
\begin{enumerate}
\item 对于隐函数$F\left(x,y\right)=0$,它的导数为$\frac{dy}{dx}=-\frac{F_x}{F_y}$
\item 对于隐函数$F\left(x,y,z\right)=0$,它的偏导数分别为$\frac{\partial z}{\partial x}=-\frac{F_x}{F_z}$和$\frac{\partial z}{\partial y}=-\frac{F_y}{F_z}$
\end{enumerate}

\subsection{几何应用}
\begin{enumerate}
\item 空间曲线的切线和法平面:设空间曲线$\Gamma$的参数方程为
\[\left\{\begin{aligned}&x=\varphi\left(t\right)\\&y=\psi\left(t\right)\\&z=\omega\left(t\right)\end{aligned}\right.,t\in\left[\alpha,\beta\right]\]
设点$M$对应的参数为$t_0$,则向量$\vec{v}=\left(\varphi^{\prime}\left(t_0\right),\psi^{\prime}\left(t_0\right),\omega^{\prime}\left(t_0\right)\right)$即为点$M$处的一个切向量,曲线$\Gamma$在该点的切线方程为:
\[\frac{x-x_0}{\varphi^{\prime}\left(t_0\right)}=\frac{y-y_0}{\psi^{\prime}\left(t_0\right)}=\frac{z-z_0}{\omega^{\prime}\left(t_0\right)}\]
法平面方程为:
\[\varphi^{\prime}\left(t_0\right)\left(x-x_0\right)+\psi^{\prime}\left(t_0\right)\left(y-y_0\right)+\omega^{\prime}\left(t_0\right)\left(z-z_0\right)=0\]
\item 空间曲面的切平面和法线:设空间曲面的方程为$F\left(x,y,z\right)=0$,设曲面上一点为$M=\left(x_0,y_0,z_0\right)$,则向量$\vec{n}=\left(F_x\left(x_0,y_0,z_0\right),F_y\left(x_0,y_0,z_0\right),F_z\left(x_0,y_0,z_0\right)\right)$即为曲面在点$M$处的法向量,曲面在该点的切平面方程为:
\[F_x\left(x_0,y_0,z_0\right)\left(x-x_0\right)+F_y\left(x_0,y_0,z_0\right)\left(y-y_0\right)+F_z\left(x_0,y_0,z_0\right)\left(z-z_0\right)=0\]
法线方程为:
\[\frac{x-x_0}{F_x\left(x_0,y_0,z_0\right)}=\frac{y-y_0}{F_y\left(x_0,y_0,z_0\right)}=\frac{z-z_0}{F_z\left(x_0,y_0,z_0\right)}\]
\end{enumerate}

\subsection{方向导数与梯度}
\begin{enumerate}
\item 方向导数:设一个方向$l$的单位向量为$\vec{e_l}=\left(\cos\alpha,\cos\beta\right)$,其中$\cos\alpha$,$\cos\beta$为方向$l$的方向余弦,则在点$P_0=\left(x_0,y_0\right)$处的方向导数为:
\[\left.\frac{\partial f}{\partial l}\right|_{\left(x_0,y_0\right)}=\lim\limits_{t\to0^{+}}\frac{f\left(x_0+t\cos\alpha,y_0+t\cos\beta\right)-f\left(x_0,y_0\right)}{t}=f_x\left(x_0,y_0\right)\cos\alpha+f_y\left(x_0,y_0\right)\cos\beta\]
\item 梯度:向量在点$P_0=\left(x_0,y_0\right)$处的梯度定义为:
\[\mathbf{grad}f\left(x_0,y_0\right)=f_x\left(x_0,y_0\right)\vec{i}+f_y\left(x_0,y_0\right)\vec{j}\]
梯度方向为函数增加最快的方向
\end{enumerate}

\subsection{多元函数的极值与条件极值}
\begin{enumerate}
\item 极值:若函数$z=f\left(x,y\right)$在点$\left(x_0,y_0\right)$处有极值,则在该点偏导数全为零,即有$f_x\left(x_0,y_0\right)=0$和$f_y\left(x_0,y_0\right)=0$;而若函数$z=f\left(x,y\right)$在点$\left(x_0,y_0\right)$处满足$f_x\left(x_0,y_0\right)=0$和$f_y\left(x_0,y_0\right)=0$,假设在该点某邻域内函数的二阶偏导数存在且连续,且假设$f_{xx}\left(x_0,y_0\right)=A$、$f_{xy}\left(x_0,y_0\right)=B$、$f_{yy}\left(x_0,y_0\right)=C$,则我们又有:
\begin{enumerate}
\item $AC-B^2>0$时具有极值,且$A<0$时为极大值,$A>0$时为极小值;
\item $AC-B^2<0$时不具有极值;
\item $AC-B^2=0$时可能具有极值也可能不具有极值,此时需要另作讨论
\end{enumerate}
\item 条件极值:若要求函数$u=f\left(x,y,z\right)$在约束条件$\varphi\left(x,y,z\right)=0$和$\psi\left(x,y,z\right)=0$下的条件极值,可以使用拉格朗日乘数法求解,求解步骤如下:
\begin{enumerate}
\item 构造拉格朗日函数:
\[F\left(x,y,z,\lambda,\mu\right)=f\left(x,y,z\right)+\lambda\varphi\left(x,y,z\right)+\mu\psi\left(x,y,z\right)\]
其中$\lambda$、$\mu$称为拉格朗日乘数
\item 构造下列方程组并求解:
\[\left\{\begin{aligned}
&F_x^{\prime}=f_x^{\prime}+\lambda\varphi_x^{\prime}+\mu\psi_x^{\prime}=0\\
&F_y^{\prime}=f_y^{\prime}+\lambda\varphi_y^{\prime}+\mu\psi_y^{\prime}=0\\
&F_z^{\prime}=f_z^{\prime}+\lambda\varphi_z^{\prime}+\mu\psi_z^{\prime}=0\\
&F_{\lambda}^{\prime}=\varphi\left(x,y,z\right)=0\\
&F_{\mu}^{\prime}=\psi\left(x,y,z\right)=0
\end{aligned}\right.\]
方程所得到的所有解即为可能的极值点,由这些可能的极值点可求得目标极值点
\end{enumerate}
\end{enumerate}


\section{重积分}
\subsection{重积分中值定理}
\begin{enumerate}
\item 二重积分中值定理:设函数$f\left(x,y\right)$在闭区域$D$上连续,$\sigma$是区域$D$的面积,则在$D$上至少存在一点$\left(\xi,\eta\right)$,使得
\[\iint\limits_{D}f\left(x,y\right)d\sigma=f\left(\xi,\eta\right)\sigma\]
\item 推广的二重积分中值定理:设函数$f\left(x,y\right)$和$g\left(x,y\right)$都在有界闭区域$D$上连续,且$g\left(x,y\right)$区域$D$上不变号,则在$D$上至少存在一点$\left(\xi,\eta\right)$,使得
\[\iint\limits_{D}f\left(x,y\right)g\left(x,y\right)d\sigma=f\left(\xi,\eta\right)\iint\limits_{D}g\left(x,y\right)d\sigma\]
\item 三重积分中值定理:设函数$f\left(x,y,z\right)$在闭区域$\Omega$上连续,$V$是区域$\Omega$的体积,则在$D$上至少存在一点$\left(\alpha,\beta,\gamma\right)$,使得
\[\iiint\limits_{\Omega}f\left(x,y,z\right)dv=f\left(\alpha,\beta,\gamma\right)V\]
\end{enumerate}

\subsection{重积分不同坐标下的计算}
\begin{enumerate}
\item 二重积分的极坐标计算:$x=r\cos\theta$,$y=r\sin\theta$,此时二重积分可化为:
\[\iint\limits_{D}f\left(x,y\right)d\sigma=\iint\limits_{D}f\left(r\cos\theta,r\sin\theta\right)rdrd\theta\]
\item 三重积分的柱坐标计算:$x=r\cos\theta$,$y=r\sin\theta$,$z=z$,此时三重积分可化为:
\[\iiint\limits_{\Omega}f\left(x,y,z\right)dv=\iiint\limits_{\Omega}f\left(r\cos\theta,r\sin\theta,z\right)rdrd\theta dz\]
\item 三重积分的球坐标计算:$x=r\sin\varphi\cos\theta$,$y=r\sin\varphi\sin\theta$,$z=r\cos\varphi$,此时三重积分可化为:
\[\iiint\limits_{\Omega}f\left(x,y,z\right)dv=\iiint\limits_{\Omega}f\left(r\sin\varphi\cos\theta,r\sin\varphi\sin\theta,r\cos\varphi\right)r^2\sin\varphi drd\varphi d\theta\]
\end{enumerate}

\subsection{重积分换元法}
\begin{enumerate}
\item 二重积分换元法:设$f\left(x,y\right)$在平面$xOy$上的闭区域$D$上连续,变换
\[T:x=x\left(u,v\right),y=y\left(u,v\right)\]
将平面$uOv$上的闭区域$D^{\prime}$变为平面$xOy$上的区域$D$,且满足下列条件:
\begin{enumerate}
\item $x\left(u,v\right)$、$y\left(u,v\right)$在$D^{\prime}$上具有一阶连续偏导数
\item 在$D^{\prime}$上雅可比行列式
\[J\left(u,v\right)=\frac{\partial\left(x,y\right)}{\partial\left(u,v\right)}=\begin{vmatrix}\frac{\partial x}{\partial u}&\frac{\partial x}{\partial v}\\\frac{\partial y}{\partial u}&\frac{\partial y}{\partial v}\end{vmatrix}\ne0\]
\item 变换$T:D^{\prime}\to D$是一一对应的
\end{enumerate}
则此时我们有:
\[\iint\limits_{D}f\left(x,y\right)dxdy=\iint\limits_{D^{\prime}}f\left[x\left(u,v\right),y\left(u,v\right)\right]\left|J\left(u,v\right)\right|dudv\]
\item 三重积分换元法:与二重积分类似,三重积分也有换元法,对于变换$T:x=x\left(u,v,w\right),y=y\left(u,v,w\right),z=z\left(u,v,w\right)$,且此时的雅可比行列式为:
\[J=\frac{\partial\left(x,y,z\right)}{\partial\left(u,v,w\right)}=\begin{vmatrix}\frac{\partial x}{\partial u}&\frac{\partial x}{\partial v}&\frac{\partial x}{\partial w}\\\frac{\partial y}{\partial u}&\frac{\partial y}{\partial v}&\frac{\partial y}{\partial w}\\\frac{\partial z}{\partial u}&\frac{\partial z}{\partial v}&\frac{\partial z}{\partial w}\end{vmatrix}\ne0\]
此时三重积分可化为:
\[\iiint\limits_{\Omega}f\left(x,y,z\right)dxdydz=\iiint\limits_{\Omega^{\prime}}f\left[x\left(u,v,w\right),y\left(u,v,w\right),z\left(u,v,w\right)\right]\left|J\right|dudvdw\]
\end{enumerate}

\subsection{几个对称性定理}
\begin{enumerate}
\item \textbf{二重积分}:
\begin{enumerate}
\item 积分区域关于某一坐标轴对称:若$f\left(x,y\right)$在积分区域$D$上连续,且$D$关于$y$轴(或$x$轴)对称,则:
\begin{enumerate}
\item 当$f\left(x,y\right)$是$D$上关于另一变量$x$(或另一变量$y$)的奇函数时,有
\[\iint\limits_{D}f\left(x,y\right)dxdy=0\]
\item 当$f\left(x,y\right)$是$D$上关于另一变量$x$(或另一变量$y$)的偶函数时,有
\[\iint\limits_{D}f\left(x,y\right)dxdy=2\iint\limits_{D_1}f\left(x,y\right)dxdy\]
其中$D_1$是$D$落在$y$轴(或$x$轴)一侧的那一部分区域
\end{enumerate}
\item 积分区域关于原点对称:设积分区域$D$对称于原点,对称于原点的两部分区域记为$D_1$和$D_2$,若$f\left(-x,-y\right)=f\left(x,y\right)$,则:
\[\iint\limits_{D}f\left(x,y\right)dxdy=2\iint\limits_{D_1}f\left(x,y\right)dxdy\]
若$f\left(-x,-y\right)=-f\left(x,y\right)$,则:
\[\iint\limits_{D}f\left(x,y\right)dxdy=0\]
\item 积分区域关于直线$y=x$对称:设积分区域$D$关于直线$y=x$对称,则此时两个变量可以交换位置,即:
\[\iint\limits_{D}f\left(x,y\right)dxdy=\iint\limits_{D}f\left(y,x\right)dxdy\]
进一步地,若将积分区域$D$沿直线$y=x$划分为对称的两个区域$D_1$、$D_2$,则此时我们有:
\[\iint\limits_{D_1}f\left(x,y\right)dxdy=\iint\limits_{D_2}f\left(y,x\right)dxdy\]
\end{enumerate}
\item \textbf{三重积分}:
\begin{enumerate}
\item 积分区域关于某一坐标面对称:若$f\left(x,y,z\right)$在有界闭区域$\Omega$上连续,且$\Omega$关于坐标平面$yOz$(或$xOy$,或$xOz$)对称,则:
\begin{enumerate}
\item 当$f\left(x,y,z\right)$是$\Omega$上关于另一变量$x$(或$z$,或$y$)的奇函数时,有
\[\iiint\limits_{\Omega}f\left(x,y,z\right)dxdydz=0\]
\item 当$f\left(x,y,z\right)$是$\Omega$上关于另一变量$x$(或$z$,或$y$)的偶函数时,有
\[\iiint\limits_{\Omega}f\left(x,y,z\right)dxdydz=2\iiint\limits_{\Omega_1}f\left(x,y,z\right)dxdydz\]
其中$\Omega_1$是$\Omega$落在$x$轴(或$z$轴,或$y$轴)为正的那一部分区域
\end{enumerate}
\item 积分区域关于某一坐标轴对称:若$f\left(x,y,z\right)$在有界闭区域$\Omega$上连续,且$\Omega$关于$x$轴(或$y$轴,或$z$轴)对称,则:
\begin{enumerate}
\item 当$f\left(x,y,z\right)$是$D$上关于另外两变量$y$、$z$(或$x$、$z$,或$x$、$y$)的奇函数时,即
\[f\left(x,-y,-z\right)=-f\left(x,y,z\right)\]
(或$f\left(-x,y,-z\right)=-f\left(x,y,z\right)$,或$f\left(-x,-y,z\right)=-f\left(x,y,z\right)$),我们有:
\[\iiint\limits_{\Omega}f\left(x,y,z\right)dxdydz=0\]
\item 当$f\left(x,y,z\right)$是$D$上关于另外两变量$y$、$z$(或$x$、$z$,或$x$、$y$)的偶函数时,即
\[f\left(x,-y,-z\right)=f\left(x,y,z\right)\]
(或$f\left(-x,y,-z\right)=f\left(x,y,z\right)$,或$f\left(-x,-y,z\right)=f\left(x,y,z\right)$),我们有:
\[\iiint\limits_{\Omega}f\left(x,y,z\right)dxdydz=2\iiint\limits_{\Omega_1}f\left(x,y,z\right)dxdydz\]
其中$\Omega_1$是$\Omega$落在对称区间靠正侧的那一半区域
\end{enumerate}
\item 积分区域关于原点对称:设积分区域$\Omega$对称于原点,对称于原点的两部分区域记为$\Omega_1$和$\Omega_2$,若$f\left(-x,-y,-z\right)=f\left(x,y,z\right)$,则:
\[\iiint\limits_{\Omega}f\left(x,y,z\right)dxdydz=2\iiint\limits_{\Omega_1}f\left(x,y,z\right)dxdydz\]
若$f\left(-x,-y,-z\right)=-f\left(x,y,z\right)$,则:
\[\iiint\limits_{\Omega}f\left(x,y,z\right)dxdydz=0\]
\end{enumerate}
\end{enumerate}

\subsection{几何应用}
\begin{enumerate}
\item \textbf{求解质心(形心)}:设平面薄片$D$的密度函数为$\mu\left(x,y\right)$,其质心为$\left(\bar{x},\bar{y}\right)$,则我们有:
\[\left\{\begin{aligned}
&\bar{x}=\frac{\iint\limits_{D}x\cdot\mu\left(x,y\right)dxdy}{\iint\limits_{D}\mu\left(x,y\right)dxdy}\\
&\bar{y}=\frac{\iint\limits_{D}y\cdot\mu\left(x,y\right)dxdy}{\iint\limits_{D}\mu\left(x,y\right)dxdy}
\end{aligned}\right.\]
当密度函数为常数时的质心就仅与物体形状有关,此时质心称为形心
\end{enumerate}


\section{曲线积分与曲面积分}
\subsection{两类曲线曲面积分计算公式}
\begin{enumerate}
\item 第一类曲线积分(对弧长的曲线积分):设$f\left(x,y\right)$在曲线弧$L$上有定义且连续,$L$的参数方程为:
\[\left\{\begin{aligned}
&x=\varphi\left(t\right)\\
&y=\psi\left(t\right)
\end{aligned}\right.,\quad\left(\alpha\leq t\leq\beta\right)\]
若$\varphi\left(t\right)$、$\psi\left(t\right)$在$\left[\alpha,\beta\right]$上具有一阶导数,且$\varphi^{\prime^2}\left(t\right)+\psi^{\prime^2}\left(t\right)\ne0$,则曲线积分$\int_{L}f\left(x,y\right)ds$存在,且:
\[\int_{L}f\left(x,y\right)ds=\int_{\alpha}^{\beta}f\left[\varphi\left(t\right),\psi\left(t\right)\right]\sqrt{\varphi^{\prime^2}\left(t\right)+\psi^{\prime^2}\left(t\right)}dt\]
\item 第二类曲线积分(对坐标的曲线积分):对坐标的曲线积分起源于变力$\vec{F}\left(x,y\right)=P\left(x,y\right)\vec{i}+Q\left(x,y\right)\vec{j}$沿曲线$L$做功问题。设$P\left(x,y\right)$、$Q\left(x,y\right)$在有向弧线$L$上有定义且连续,$L$的参数方程为:
\[\left\{\begin{aligned}
&x=\varphi\left(t\right)\\
&y=\psi\left(t\right)
\end{aligned}\right.\]
当参数$t$单调地从$\alpha$变到$\beta$时,点$M\left(x,y\right)$从$L$的起点$A$沿$L$运动到终点$B$,若$\varphi\left(t\right)$与$\psi\left(t\right)$在以$\alpha$及$\beta$为端点的闭区间上具有一阶连续导数,且$\varphi^{\prime^2}\left(t\right)+\psi^{\prime^2}\left(t\right)\ne0$,则曲线积分$\int_{L}P\left(x,y\right)dx+Q\left(x,y\right)dy$存在,且:
\[\int_{L}P\left(x,y\right)dx+Q\left(x,y\right)dy=\int_{\alpha}^{\beta}\left\{P\left[\varphi\left(t\right),\psi\left(t\right)\right]\varphi^{\prime}\left(t\right)+Q\left[\varphi\left(t\right),\psi\left(t\right)\right]\psi^{\prime}\left(t\right)\right\}dt\]
\item 两类曲线积分的联系:$\int_{L}P\left(x,y\right)dx+Q\left(x,y\right)dy=\int_{L}\left(P\cos\alpha+Q\cos\beta\right)ds$,其中$\alpha\left(x,y\right)$与$\beta\left(x,y\right)$为有向线段弧$L$在点$\left(x,y\right)$处的切向量的方向角
\item 第一类曲面积分(对面积的曲面积分):设积分曲面$\Sigma$由方程$z=z\left(x,y\right)$给出,$\Sigma$在$xOy$面上的投影区域为$D_{xy}$,函数$z=\left(x,y\right)$在$D_{xy}$上具有连续偏导数,被积函数$f\left(x,y,z\right)$在$\Sigma$上连续,则曲面积分$\iint\limits_{\Sigma}f\left(x,y,z\right)dS$存在,且:
\[\iint\limits_{\Sigma}f\left(x,y,z\right)dS=\iint\limits_{D_{xy}}f\left[x,y,z\left(x,y\right)\right]\sqrt{1+z_x^2\left(x,y\right)+z_y^2\left(x,y\right)}dxdy\]
若积分曲面$\Sigma$由方程$x=x\left(y,z\right)$或$y=\left(z,x\right)$给出,也可类似的将曲面积分转化为二重积分
\item 第二类曲面积分(对坐标的曲面积分):对坐标的曲面积分起源于对流向曲面$\Sigma$一侧的具有速度$\vec{v}\left(x,y,z\right)=P\left(x,y,z\right)\vec{i}+Q\left(x,y,z\right)\vec{j}+R\left(x,y,z\right)\vec{k}$的不可压缩流体的流量的计算。设积分区域$\Sigma$是由方程$z=z\left(x,y\right)$所给出的曲面上侧,$\Sigma$在$xOy$面上的投影区域为$D_{xy}$,函数$z=\left(x,y\right)$在$D_{xy}$上具有一阶连续偏导数,被积函数$R\left(x,y,z\right)$在$\Sigma$上连续,则有:
\[\iint\limits_{\Sigma}R\left(x,y,z\right)dxdy=\pm\iint\limits_{D_{xy}}R\left[x,y,z\left(x,y\right)\right]dxdy\]
此时曲面$\Sigma$的法向量与坐标轴$z$轴正向的夹角小于$\frac{\pi}{2}$时取正号,大于$\frac{\pi}{2}$时取负号\\
类似的,如果$\Sigma$由$x=x\left(y,z\right)$给出,那么有
\[\iint\limits_{\Sigma}P\left(x,y,z\right)dydz=\pm\iint\limits_{D_{yz}}P\left[x\left(y,z\right),y,z\right]dydz\]
此时曲面$\Sigma$的法向量与坐标轴$x$轴正向的夹角小于$\frac{\pi}{2}$时取正号,大于$\frac{\pi}{2}$时取负号\\
如果$\Sigma$由$y=y\left(z,x\right)$给出,那么有
\[\iint\limits_{\Sigma}Q\left(x,y,z\right)dzdx=\pm\iint\limits_{D_{zx}}Q\left[x,y\left(z,x\right),z\right]dzdx\]
此时曲面$\Sigma$的法向量与坐标轴$y$轴正向的夹角小于$\frac{\pi}{2}$时取正号,大于$\frac{\pi}{2}$时取负号
\item 两类曲面积分的联系:$\iint\limits_{\Sigma}Pdydz+Qdzdx+Rdxdy=\iint\limits_{\Sigma}\left(P\cos\alpha+Q\cos\beta+R\cos\gamma\right)dS$,其中$\cos\alpha$、$\cos\beta$与$\cos\gamma$是有向曲面$\Sigma$在点$\left(x,y,z\right)$处的法向量的方向余弦
\end{enumerate}

\subsection{格林公式}
\begin{enumerate}
\item 正方向:对平面区域$D$的边界曲线$L$,我们规定,称一个方向为$L$的正方向,仅当观察者沿着$L$的这个方向行走时区域$D$总是在观察者左手边,反之则为负方向
\item 格林公式:设闭区域$D$由分段光滑的曲线$L$围成,若函数$P\left(x,y\right)$与$Q\left(x,y\right)$在区域$D$上有一阶连续偏导数,则有
\[\iint\limits_{D}\left(\frac{\partial Q}{\partial x}-\frac{\partial P}{\partial y}\right)dxdy=\oint_{L}Pdx+Qdy\]
其中$L$是$D$的取正向的边界曲线
\item (第二类)曲线积分与路径无关的条件:设区域$G$是一个单连通域,若函数$P\left(x,y\right)$与$Q\left(x,y\right)$在区域$G$上有一阶连续偏导数,则曲线积分$\int_{L}P\left(x,y\right)dx+Q\left(x,y\right)dy$在$G$内与路径无关(或沿$G$内任意闭曲线的曲线积分为零)的充分必要条件为
\[\frac{\partial P}{\partial y}=\frac{\partial Q}{\partial x}\]
在$G$内恒成立。\\
此外,由于$P\left(x,y\right)dx+Q\left(x,y\right)dy$在区域$G$内为某一函数$u\left(x,y\right)$的全微分的充分必要条件也是$\frac{\partial P}{\partial y}=\frac{\partial Q}{\partial x}$在$G$内恒成立,故曲线积分与路径无关的条件也可以描述为:第二类曲线积分$\int_{L}P\left(x,y\right)dx+Q\left(x,y\right)dy$在$G$内与路径无关(或沿$G$内任意闭曲线的曲线积分为零)的充分必要条件为在$G$内存在函数$u\left(x,y\right)$,使得$du=Pdx+Qdy$,故当第二类曲线积分与路径无关时也可以利用牛顿-莱布尼兹公式求解积分,即假设积分沿某光滑曲线$L$从点$A\left(x_a,y_a\right)$积分到$B\left(x_b,y_b\right)$,且在区域$G$内存在函数$u\left(x,y\right)$,使得$du=Pdx+Qdy$,此时我们有
\[\int_{L}P\left(x,y\right)dx+Q\left(x,y\right)dy=u\left(x_b,y_b\right)-u\left(x_a,y_a\right)\]
\end{enumerate}

\subsection{高斯公式}
\begin{enumerate}
\item 高斯公式:设空间闭区域$\Omega$是由分片光滑的闭曲面$\Sigma$所围成,若函数$P\left(x,y,z\right)$、$Q\left(x,y,z\right)$以及$R\left(x,y,z\right)$在$\Omega$上有一阶连续偏导数,则有
\[\iiint\limits_{\Omega}\left(\frac{\partial P}{\partial x}+\frac{\partial Q}{\partial y}+\frac{\partial R}{\partial z}\right)dv=\oiint\limits_{\Sigma}Pdydz+Qdzdx+Rdxdy\]
其中$\Sigma$是$\Omega$的整个边界曲面的外侧
\item (第二类)曲面积分与曲面$\Sigma$无关而仅与$\Sigma$的边界曲线有关的条件:设$G$是二维单连通区域,若$P\left(x,y,z\right)$、$Q\left(x,y,z\right)$与$R\left(x,y,z\right)$在$G$上有一阶连续偏导数,则曲面积分$\iint\limits_{\Sigma}Pdydz+Qdzdx+Rdxdy$在$G$内与所选曲面$\Sigma$无关而仅与$\Sigma$的边界曲线有关(或沿$G$内任一闭曲面的曲面积分为零)的充分必要条件是
\[\frac{\partial P}{\partial x}+\frac{\partial Q}{\partial y}+\frac{\partial R}{\partial z}=0\]
在$G$内恒成立
\end{enumerate}

\subsection{斯托克斯公式}
\begin{enumerate}
\item 斯托克斯公式:设$\Gamma$为分段光滑的空间有向闭曲线,$\Sigma$是以$\Gamma$为边界的分片光滑的有向曲面,$\Gamma$的正向与$\Sigma$的侧符合右手规则(即当右手除大拇指外的四指依曲线$\Gamma$的正向绕行时,竖起的大拇指的指向应当与曲面$\Sigma$的法向量指向一致),若函数$P\left(x,y,z\right)$、$Q\left(x,y,z\right)$与 $R\left(x,y,z\right)$在曲面$\Sigma$(连同边界$\Gamma$)上有一阶连续偏导数,则有
\[\iint\limits_{\Sigma}\left(\frac{\partial R}{\partial y}-\frac{\partial Q}{\partial z}\right)dydz+\left(\frac{\partial P}{\partial z}-\frac{\partial R}{\partial x}\right)dzdx+\left(\frac{\partial Q}{\partial x}-\frac{\partial P}{\partial y}\right)dxdy=\oint_{\Gamma}Pdx+Qdy+Rdz\]
当$\Sigma$是$xOy$平面上的一块平面闭区域时,斯托克斯公式便成为了格林公式
\item 空间(第二类)曲线积分与路径无关的条件:设区域$G$是一维单连通域,若函数$P\left(x,y,z\right)$、$Q\left(x,y,z\right)$与$R\left(x,y,z\right)$在区域$G$上有一阶连续偏导数,则曲线积分$\int_{\Gamma}Pdx+Qdy+Rdz$在$G$内与路径无关(或沿$G$内任意闭曲线的曲线积分为零)的充分必要条件为
\[\frac{\partial P}{\partial y}=\frac{\partial Q}{\partial x},\frac{\partial Q}{\partial z}=\frac{\partial R}{\partial y},\frac{\partial R}{\partial x}=\frac{\partial P}{\partial z}\]
在$G$内恒成立
\end{enumerate}

\subsection{通量与散度、环流量与旋度}
\begin{enumerate}
\item 通量:设有向量场$\vec{A}\left(x,y,z\right)=P\left(x,y,z\right)\vec{i}+Q\left(x,y,z\right)\vec{j}+R\left(x,y,z\right)\vec{k}$,其中$P$、$Q$和$R$均具有一阶连续偏导数,$\Sigma$是场内的一片有向曲面,$\vec{n}$是$\Sigma$在点$\left(x,y,z\right)$处的单位法向量,则积分
\[\iint\limits_{\Sigma}\vec{A}\cdot\vec{n}dS=\iint\limits_{\Sigma}\vec{A}\cdot d\vec{S}=\iint\limits_{\Sigma}Pdydz+Qdzdx+Rdxdy\]
称为向量场$\vec{A}$通过曲面$\Sigma$向着指定侧的通量
\item 散度(又称通量密度):设在闭区域$\Omega$上具有速度场为$\vec{v}\left(x,y,z\right)=P\left(x,y,z\right)\vec{i}+Q\left(x,y,z\right)\vec{j}+R\left(x,y,z\right)\vec{k}$的不可压缩流体,则在$\Omega$上某一点$M\left(x,y,z\right)$的散度为
\[\operatorname{div}\vec{v}\left(M\right)=\frac{\partial P}{\partial x}+\frac{\partial Q}{\partial y}+\frac{\partial R}{\partial z}\]
\item 环流量:设有向量场$\vec{A}\left(x,y,z\right)=P\left(x,y,z\right)\vec{i}+Q\left(x,y,z\right)\vec{j}+R\left(x,y,z\right)\vec{k}$,其中$P$、$Q$和$R$均连续,$\Gamma$是$\vec{A}$的定义域内的一条分段光滑的有向闭曲线,$\vec{\sigma}$是$\Gamma$在点$\left(x,y,z\right)$处的单位切向量,则积分
\[\oint\limits_{\Gamma}\vec{A}\cdot\vec{\sigma}ds=\oint\limits_{\Gamma}\vec{A}\cdot d\vec{r}=\oint\limits_{\Gamma}Pdx+Qdy+Rdz\]
称为向量场$\vec{A}$沿有向闭曲线$\Gamma$的环流量
\item 旋度(又称环量密度):设有向量场$\vec{A}\left(x,y,z\right)=P\left(x,y,z\right)\vec{i}+Q\left(x,y,z\right)\vec{j}+R\left(x,y,z\right)\vec{k}$,其中$P$、$Q$和$R$均具有一阶连续偏导数,则向量场$\vec{A}$的旋度为
\[\operatorname{rot}\vec{A}=\left(\frac{\partial R}{\partial y}-\frac{\partial Q}{\partial z}\right)\vec{i}+\left(\frac{\partial P}{\partial z}-\frac{\partial R}{\partial x}\right)\vec{j}+\left(\frac{\partial Q}{\partial x}-\frac{\partial P}{\partial y}\right)\vec{k}\]
\end{enumerate}

\subsection{对称性化简积分}
\begin{enumerate}
\item 第一类曲线积分:
\begin{enumerate}
\item 设曲线$L$关于$y$轴对称,则
\[\int_{L}f\left(x,y\right)ds=
\left\{\begin{aligned}
2\int_{L_1}f\left(x,y\right)ds\quad&\left(f\left(x,y\right)\text{关于变量}x\text{是偶函数}\right)\\
0\qquad&\left(f\left(x,y\right)\text{关于变量}x\text{是奇函数}\right)
\end{aligned}\right.\]
其中$L_1$是$L$在$x\ge0$的那段曲线\\
设曲线$L$关于$x$轴对称,则
\[\int_{L}f\left(x,y\right)ds=
\left\{\begin{aligned}
2\int_{L_1}f\left(x,y\right)ds\quad&\left(f\left(x,y\right)\text{关于变量}y\text{是偶函数}\right)\\
0\qquad&\left(f\left(x,y\right)\text{关于变量}y\text{是奇函数}\right)
\end{aligned}\right.\]
其中$L_1$是$L$在$y\ge0$的那段曲线
\item 设$f\left(x,y\right)$在分段光滑曲线$L$上连续,若$L$关于原点对称,则
\[\int_{L}f\left(x,y\right)ds=
\left\{\begin{aligned}
2\int_{L_1}f\left(x,y\right)ds\quad&\left(f\left(x,y\right)\text{关于}\left(x,y\right)\text{是偶函数}\right)\\
0\qquad&\left(f\left(x,y\right)\text{关于}\left(x,y\right)\text{是奇函数}\right)
\end{aligned}\right.\]
其中$L_1$是$L$的右半平面或上半平面部分
\item 若$L$关于直线$y=x$对称,则$\int_{L}f\left(x,y\right)ds=\int_{L}f\left(y,x\right)ds$
\item 若曲线$\Gamma$方程中三变量$x$、$y$、$z$具有“对等”性质,即$x$、$y$、$z$三个变量中的任意两个对换,$\Gamma$的方程不变(即$x$、$y$、$z$具有轮换对称性),则有
\[\oint_{\Gamma}xds=\oint_{\Gamma}yds=\oint_{\Gamma}zds,\quad\oint_{\Gamma}x^2ds=\oint_{\Gamma}y^2ds=\oint_{\Gamma}z^2ds\]
\end{enumerate}
\item 第二类曲线积分:\\
设$L$为平面上分段光滑的定向曲线,$P\left(x,y\right)$、$Q\left(x,y\right)$连续
\begin{enumerate}
\item $L$关于$y$轴对称,则
\[\int_{L}Pdx=\left\{\begin{aligned}
2\int_{L_1}Pdx\quad&\left(P\text{是关于另一变量}x\text{的偶函数}\right)\\
0\qquad&\left(P\text{是关于另一变量}x\text{的奇函数}\right)
\end{aligned}\right.\]
\[\int_{L}Qdy=\left\{\begin{aligned}
0\qquad&\left(Q\text{是关于另一变量}x\text{的偶函数}\right)\\
2\int_{L_1}Qdy\quad&\left(Q\text{是关于另一变量}x\text{的奇函数}\right)
\end{aligned}\right.\]
其中$L_1$是$L$在右半平面的部分
\item $L$关于$x$轴对称,则
\[\int_{L}Pdx=\left\{\begin{aligned}
0\qquad&\left(P\text{是关于另一变量}y\text{的偶函数}\right)\\
2\int_{L_1}Pdx\quad&\left(P\text{是关于另一变量}y\text{的奇函数}\right)
\end{aligned}\right.\]
\[\int_{L}Qdy=\left\{\begin{aligned}
2\int_{L_1}Qdy\quad&\left(Q\text{是关于另一变量}y\text{的偶函数}\right)\\
0\qquad&\left(Q\text{是关于另一变量}y\text{的奇函数}\right)
\end{aligned}\right.\]
其中$L_1$是$L$在上半平面的部分
\item $L$关于原点对称,则
\[\int_{L}Pdx+Qdy=\left\{\begin{aligned}
0\qquad&\left(P\text{、}Q\text{是关于}\left(x,y\right)\text{的偶函数}\right)\\
2\int_{L_1}Pdx+Qdy\quad&\left(P\text{、}Q\text{是关于}\left(x,y\right)\text{的奇函数}\right)
\end{aligned}\right.\]
其中$L_1$是$L$在右半平面或上半平面的部分
\end{enumerate}
\end{enumerate}


\section{无穷级数}
\subsection{级数收敛}
\begin{enumerate}
\item \textbf{级数收敛定义}:设无穷数列$u_{k}$前$n$项和为$S_n=u_1+u_2+\cdots+u_n$,称$S_n$为级数的部分和,若极限$\lim\limits_{n\to\infty}S_n=S$,则称级数收敛,且有$\sum\limits_{n=1}^{\infty}u_n=S$
\item \textbf{绝对收敛}:对任意项级数$\sum\limits_{n=1}^{\infty}u_n$,若$\sum\limits_{n=1}^{\infty}\left|u_n\right|$收敛,则称原级数$\sum\limits_{n=1}^{\infty}u_n$绝对收敛。级数绝对收敛则原级数一定收敛
\item \textbf{条件收敛}:若原级数$\sum\limits_{n=1}^{\infty}u_n$收敛,但取绝对值以后级数$\sum\limits_{n=1}^{\infty}\left|u_n\right|$发散,则称原级数条件收敛
\end{enumerate}

\subsection{常数项级数审敛法}
\begin{enumerate}
\item 正项级数审敛法:
\begin{enumerate}
\item 比较审敛法极限形式:设$\sum\limits_{n=1}^{\infty}u_n$和$\sum\limits_{n=1}^{\infty}v_n$都是正项级数,则
\begin{enumerate}
\item 若$\lim\limits_{n\to\infty}\frac{u_n}{v_n}=l\left(0\leq l<+\infty\right)$,且级数$\sum\limits_{n=1}^{\infty}v_n$收敛,那么级数$\sum\limits_{n=1}^{\infty}u_n$收敛
\item 若$\lim\limits_{n\to\infty}\frac{u_n}{v_n}=l>0$或$\lim\limits_{n\to\infty}\frac{u_n}{v_n}=\infty$,且级数$\sum\limits_{n=1}^{\infty}v_n$发散,那么级数$\sum\limits_{n=1}^{\infty}u_n$发散
\end{enumerate}
\item 比值审敛法:设$\sum\limits_{n=1}^{\infty}u_n$为正项级数,如果
\[\lim\limits_{n\to\infty}\frac{u_{n+1}}{u_n}=\rho\]
则当$\rho<1$时级数收敛,当$\rho>1$或$\lim\limits_{n\to\infty}\frac{u_{n+1}}{u_n}=\infty$时级数发散,$\rho=1$时级数可能收敛也可能发散
\item 根值审敛法:设$\sum\limits_{n=1}^{\infty}u_n$为正项级数,如果
\[\lim\limits_{n\to\infty}\sqrt[n]{u_n}=\rho\]
则当$\rho<1$时级数收敛,当$\rho>1$或$\lim\limits_{n\to\infty}\sqrt[n]{u_n}=+\infty$时级数发散,$\rho=1$时级数可能收敛也可能发散
\item 极限审敛法:设$\sum\limits_{n=1}^{\infty}u_n$为正项级数,
\begin{enumerate}
\item 如果$p>1$,而$\lim\limits_{n\to\infty}n^pu_n=l\left(0\leq l<+\infty\right)$,那么级数$\sum\limits_{n=1}^{\infty}u_n$收敛
\item 如果$\lim\limits_{n\to\infty}nu_n=l>0$或$\lim\limits_{n\to\infty}nu_n=+\infty$,那么级数$\sum\limits_{n=1}^{\infty}u_n$发散
\end{enumerate}
\item 柯西积分审敛法:设$\sum\limits_{n=1}^{\infty}u_n$为正项级数,若将一般项$u_n$表达式中的离散变量$n$用连续变量$x$替换则可以得到一个函数$u\left(x\right)$,此时若函数$u\left(x\right)$在区间$\left[a,+\infty\right)$($a$为一个满足$a\ge 1$的任意值)内非负、连续且单调减少,则反常积分$\int_{a}^{+\infty}u\left(x\right)dx$与原级数$\sum\limits_{n=1}^{\infty}u_n$有相同的敛散性(对于一些通项表达式看起来很好积分的级数可以考虑使用柯西积分审敛法)
\end{enumerate}
\item 交错级数审敛法:
\begin{enumerate}
\item 定理1:对于交错级数,首先考虑其绝对收敛性,若其绝对收敛,则原级数也收敛\\
莱布尼兹定理:如果交错级数$\sum\limits_{n=1}^{\infty}\left(-1\right)^{n-1}u_n$满足条件:
\begin{enumerate}
\item $u_n\ge u_{n+1}\left(n=1,2,3,\cdots\right)$
\item $\lim\limits_{n\to\infty}u_n=0$
\end{enumerate}
那么级数收敛,且级数和$s\leq u_1$,余项$r_n$满足$\left|r_n\right|\leq u_{n+1}$
\end{enumerate}
\item 一般级数审敛法:
\begin{enumerate}
\item 命题1:对于级数$\sum\limits_{n=1}^{\infty}u_n$,若$\lim\limits_{n\to\infty}S_{2n}$与$\lim\limits_{n\to\infty}S_{2n+1}$均存在且相等,则该级数收敛,否则级数发散\\
柯西审敛原理:级数$\sum\limits_{n=1}^{\infty}u_n$收敛的充要条件为:对于任意给定的正数$\varepsilon$,总存在正整数$N$,使得当$n>N$时,对于任意的正整数$p$,都有
\[\left|u_{n+1}+u_{n+2}+\cdots+u_{u+p}\right|=\left|S_{n+p}-S_{n}\right|<\varepsilon\]
\item 命题2:对于一般级数$\sum\limits_{n=1}^{\infty}u_n$(显然对于交错级数也适用),我们有比值审敛法和根值审敛法的一般形式:
\begin{enumerate}
\item 当$\lim\limits_{n\to\infty}\frac{\left|u_{n+1}\right|}{\left|u_n\right|}=\rho<1$时级数收敛,当$\rho>1$或$\lim\limits_{n\to\infty}\frac{u_{n+1}}{u_n}=\infty$时级数发散,$\rho=1$时级数可能收敛也可能发散
\item 当$\lim\limits_{n\to\infty}\sqrt[n]{\left|u_n\right|}=\rho<1$时级数收敛,当$\rho>1$或$\lim\limits_{n\to\infty}\sqrt[n]{u_n}=+\infty$时级数发散,$\rho=1$时级数可能收敛也可能发散
\end{enumerate}
\end{enumerate}
\end{enumerate}

\subsection{幂级数}
\begin{enumerate}
\item 定理:对于一个幂级数$\sum\limits_{n=0}^{\infty}a_nx^n$,若
\[\lim\limits_{n\to\infty}\left|\frac{a_{n+1}}{a_n}\right|=\rho\left(\text{或}\lim\limits_{n\to\infty}\sqrt[n]{\left|a_n\right|}=\rho\right)\]
则该幂级数的收敛半径为:
\[R=\left\{\begin{aligned}
\frac{1}{\rho},&\quad\rho\ne0\\
+\infty,&\quad\rho=0\\
0,&\quad\rho=+\infty
\end{aligned}\right.\]
此时在区间$\left(-R,R\right)$内绝对收敛,在区间$\left(-\infty,-R\right)\cup\left(R,+\infty\right)$内发散,在$x\pm R$处可能收敛也可能发散\\
(由此可知若级数$\sum\limits_{n=1}^{\infty}a_nk^n$条件收敛,则幂级数$\sum\limits_{n=1}^{\infty}a_nx^n$的收敛半径一定为$R=k$,即幂级数只可能在收敛半径处条件收敛)
\item 性质:幂级数$\sum\limits_{n=0}^{\infty}a_nx^n$的和函数$s\left(x\right)$在其收敛半径区间$\left(-R,R\right)$内有任意阶导数和积分,且导数与积分后的收敛半径不变\\
常见的会展开为幂级数使用的函数:$e^x$、$\sin x$、$\cos x$、$\frac{1}{1-x}$、$\ln\left(1+x\right)$、$\arctan x$
\end{enumerate}

\subsection{傅里叶级数}
\begin{enumerate}
\item 狄利克雷收敛定理:设$f\left(x\right)$是周期为$2\pi$的周期函数,如果它满足:
\begin{enumerate}
\item 在一个周期内连续或只有有限个第一类间断点
\item 在一个周期内至多只有有限个极值点
\end{enumerate}
则$f\left(x\right)$的傅里叶级数收敛,且当$x$是$f\left(x\right)$的连续点时,级数收敛于$f\left(x\right)$,而当$x$是$f\left(x\right)$的间断点时,级数收敛于$\frac{1}{2}\left[f\left(x^-\right)+f\left(x^+\right)\right]$
\item 一般周期函数傅里叶展开公式:设周期为$2l$的周期函数$f\left(x\right)$满足收敛定理的条件,则它的傅里叶级数展开式为:
\[f\left(x\right)=\frac{a_0}{2}+\sum\limits_{n=1}^{\infty}\left(a_n\cos\frac{n\pi x}{l}+b_n\sin\frac{n\pi x}{l}\right)\quad\left(x\in C\right)\]
其中
\[\begin{aligned}
a_n&=\frac{1}{l}\int_{-l}^{l}f\left(x\right)\cos\frac{n\pi x}{l}dx\quad\left(n=0,1,2,\cdots\right)\\
b_n&=\frac{1}{l}\int_{-l}^{l}f\left(x\right)\sin\frac{n\pi x}{l}dx\quad\left(n=1,2,3,\cdots\right)
\end{aligned}\]
\end{enumerate}







\chapter{解题笔记}

\section{函数与极限}
\begin{enumerate}
\item 求解数列递推形式极限
\begin{enumerate}
\item 第一种题型\\
假设数列极限值为$A$,利用递推式构造等式求出极限\\
例:求下列数列极限:
\[x_1=\sqrt{2},\ x_{n+1}=\sqrt{2+x_n}\ \left(n=1,2,\cdots\right)\]
解:由数学归纳法易知数列为单调递增数列,若再假设该数列有上确界$A$,则数列极限必然存在且为$A$,现在先求$A$,由递推式可得:
\[\begin{aligned}
A&=\sqrt{2+A}\\
A^2-A-2&=0\\
A=-1&\text{或}A=2
\end{aligned}\]
由于数列单调递增且$x_1=\sqrt{2}$,则$A=2$。现在证明$A=2$确实是该数列的上界,由数学归纳法,若$x_k<2$,则有$x_{k+1}=\sqrt{2+x_k}<\sqrt{4}=2$,故$A=2$确实是数列的上界,数列确实单调有界,极限存在,递推式所求出的极限$A=2$确实为数列的极限
\item 第二种题型\\
当递推式不可构造出能求出极限的等式时,考虑其他做法\\
例:已知数列满足$x_{n+2}-\frac{4}{3}x_{n+1}+\frac{1}{3}x_n=0,\ n=1,2,\cdots$,且$x_1=1,\ x_2=2$,则$\left\{x_n\right\}$收敛于多少\\
解:
\[\begin{aligned}
x_{n+2}-\frac{4}{3}x_{n+1}+\frac{1}{3}x_n&=0\\
x_{n+2}-x_{n+1}&=\frac{1}{3}\left(x_{n+1}-x_n\right)\\
\frac{x_{n+2}-x_{n+1}}{x_{n+1}-x_n}&=\frac{1}{3}
\end{aligned}\]
则我们易知$x_{n+2}-x_{n+1}=\frac{1}{3^n}$,于是我们有
\[\begin{aligned}
&x_{n+2}-x_2\\
=&x_{n+2}-x_{n+1}+x_{n+1}-x_{n}+\cdots+x_3-x_2\\
=&\sum\limits_{i=1}^{n}\frac{1}{3^i}\\
=&\frac{\frac{1}{3}-\frac{1}{3^{n+1}}}{1-\frac{1}{3}}
\end{aligned}\]
故$x_{n+2}=2+\frac{1-\frac{1}{3^n}}{2}$,显然$\lim\limits_{n\to\infty}x_n=\frac{5}{2}$
\end{enumerate}
\end{enumerate}


\section{导数与微分}
\begin{enumerate}
\item 反函数高阶导数\\
例:设函数$y=f\left(x\right)$在$x=0$的某邻域内二阶可导,$f\left(0\right)=3,\ f^{\prime}\left(0\right)=f^{\prime\prime}\left(0\right)=\frac{1}{2}$,则$\frac{d^2x}{dy^2}|_{y=3}$等于多少\\
解:
\[\begin{aligned}
&\frac{d^2x}{dy^2}\\
=&\frac{d\left(\frac{dx}{dy}\right)}{dy}\\
=&\frac{d\left(\frac{1}{\frac{dy}{dx}}\right)}{dx}\cdot\frac{dx}{dy}\\
=&\frac{d\left(\frac{1}{f^{\prime}\left(x\right)}\right)}{dx}\cdot\frac{1}{f^{\prime}\left(x\right)}\\
=&-\frac{f^{\prime\prime}\left(x\right)}{\left(f^{\prime}\left(x\right)\right)^2}\cdot\frac{1}{f^{\prime}\left(x\right)}\\
=&-\frac{f^{\prime\prime}\left(x\right)}{\left(f^{\prime}\left(x\right)\right)^3}
\end{aligned}\]
由于在$f^{\prime}\left(0\right)=f^{\prime\prime}\left(0\right)=\frac{1}{2}>0$,在$x=0$的某邻域内单调,故当$y=3$时$x=0$,故原式答案为$-\frac{f^{\prime\prime}\left(0\right)}{\left(f^{\prime}\left(0\right)\right)^3}=-4$
\item 对数求导法\\
对数求导法适用于由乘、除、乘方、开方多次运算得到的函数,取对数后再求导可以大幅简化计算\\
例:求下列函数的导数:\\
\[y=\sqrt{x\ln x\sqrt{1-\sin x}}\]
解:\\
\[\begin{aligned}
y&=\sqrt{x\ln x\sqrt{1-\sin x}}\\
\ln y&=\frac{1}{2}\left[\ln x+\ln\ln x+\frac{1}{2}\ln\left(1-\sin x\right)\right]\\
\frac{y^{\prime}}{y}&\xlongequal{\text{等式两端求导}}\frac{1}{2}\left[\frac{1}{x}+\frac{1}{x\ln x}+\frac{-\cos x}{2-2\sin x}\right]\\
y^{\prime}&=\frac{1}{2}\sqrt{x\ln x\sqrt{1-\sin x}}\cdot\left[\frac{1}{x}+\frac{1}{x\ln x}+\frac{-\cos x}{2-2\sin x}\right]\\
\end{aligned}\]
\item 泰勒展开求解高阶导数\\
在求解函数的某高阶导数的值时(多为$f^{\left(n\right)}\left(0\right)$),有时函数不为$y=g\left(x\right)h\left(x\right)$的形式而是为其他不易直接求解高阶导数的形式(如$y=\frac{g\left(x\right)}{h\left(x\right)}$),此时可以利用泰勒展开来求解\\
例:假设函数$f\left(x\right)=\frac{x}{1-2x^4}$,若要求值$f^{\left(101\right)}\left(0\right)$,此时可将函数展开:\\
\[\begin{aligned}
f\left(x\right)&=\frac{x}{1-2x^4}\\
&=x\sum\limits_{n=0}^{\infty}\left(2x^4\right)^n\\
&=\sum\limits_{n=0}^{\infty}2^nx^{4n+1}\\
\end{aligned}\]
此时易知展开式中$x^{101}$的项为$\frac{f^{\left(101\right)}\left(0\right)}{101!}=2^{25}$,故$f^{\left(101\right)}\left(0\right)=2^{25}\cdot101!$
\end{enumerate}


\section{微分中值定理}
\begin{enumerate}
\item 等式内出现多个形如$\xi$、$\eta$的显然需要使用中值定理的变量\\
当等式中出现多个形如$\xi$、$\eta$的显然需要使用中值定理的变量时,需要考虑对一个式子使用多次中值定理来得到不同的变量,有几个变量就意味着需要几次中值定理\\
例:如果要证明式子$\left(b+a\right)f^{\prime}\left(\xi\right)=2\xi f^{\prime}\left(\eta\right)$,由于我们有$\frac{f\left(b\right)-f\left(a\right)}{b^2-a^2}=\frac{f^{\prime}\left(\xi\right)}{2\xi}$,又有$\frac{f\left(b\right)-f\left(a\right)}{b^2-a^2}=\frac{f\left(b\right)-f\left(a\right)}{\left(b-a\right)\left(b+a\right)}=\frac{f^{\prime}\left(\eta\right)}{\left(b+a\right)}$,故我们有$\left(b+a\right)f^{\prime}\left(\xi\right)=2\xi f^{\prime}\left(\eta\right)$
\end{enumerate}


\section{不定积分}
\begin{enumerate}
\item 有理函数积分\\
两个多项式的商$\frac{P\left(x\right)}{Q\left(x\right)}$称为有理函数,又称为有理分式,当有理分式不可约且分母多项式次数大于分子多项式次数时有理分式称为真分式,对于一个真分式,若分母可分解为两个多项式的乘积
\[Q\left(x\right)=Q_1\left(x\right)Q_2\left(x\right)\]
且$Q_1\left(x\right)$与$Q_2\left(x\right)$没有公因式,则原真分式可分解为
\[\frac{P\left(x\right)}{Q\left(x\right)}=\frac{P_1\left(x\right)}{Q_1\left(x\right)}+\frac{P_2\left(x\right)}{Q_2\left(x\right)}\]
利用该思路可以对积分中的有理分式进行化简\\
例:对于积分$\int{\frac{x+2}{\left(2x+1\right)\left(x^2+x+1\right)}dx}$,设$\frac{x+2}{\left(2x+1\right)\left(x^2+x+1\right)}=\frac{A}{2x+1}+\frac{Bx+C}{x^2+x+1}$,则可以得到:
\[\frac{A}{2x+1}+\frac{Bx+C}{x^2+x+1}=\frac{\left(A+2B\right)x^2+\left(A+B+2C\right)x+\left(A+C\right)}{\left(2x+1\right)\left(x^2+x+1\right)}=\frac{x+2}{\left(2x+1\right)\left(x^2+x+1\right)}\]
即我们有:
\[\left\{
\begin{aligned}
&A+2B=0\\
&A+B+2C=1\\
&A+C=2
\end{aligned}
\right.
\Rightarrow
\left\{
\begin{aligned}
&A=2\\
&B=-1\\
&C=0
\end{aligned}
\right.\]
故积分可化为
\[\int{\frac{x+2}{\left(2x+1\right)\left(x^2+x+1\right)}dx}=\int{\frac{2}{2x+1}dx}+\int{\frac{-x}{x^2+x+1}dx}\]
\item 形如$\int{\frac{a_1\sin x+b_1\cos x}{a\sin x+b\cos x}dx}$的积分方法\\
\normalsize 对于这一类积分,可以利用$\left(\sin x\right)^\prime=\cos x$及$\left(\cos x\right)^\prime=-\sin x$,对分式进行分解,即我们设:
\[a_1\sin x+b_1\cos x=A\left(a\sin x+b\cos x\right)+B\left(a\sin x+b\cos x\right)^\prime\]
此时原积分可化简为:
\[\int{\frac{a_1\sin x+b_1\cos x}{a\sin x+b\cos x}dx}=\int{Adx}+\int{\frac{B}{a\sin x+b\cos x}d\left(a\sin x+b\cos x\right)}\]
\item 万能变换\\
对于要求的积分$\int{f\left(\sin x,\cos x\right)dx}$,可以通过代换$u=\tan\left(\frac{x}{2}\right)$将其化为有理分式积分,此时有:
\[\sin x=\frac{2u}{1+u^2}, \cos x=\frac{1-u^2}{1+u^2}, dx=\frac{2}{1+u^2}du\]
\end{enumerate}


\section{定积分}
计算定积分(尤其是后续的重积分以及曲线曲面积分)时先考虑对称性及周期性,利用对称性周期性化简后再进行后续计算,其次是考虑积分的几何意义,若可以变为求易求的几何图形的面积体积等则可能可以大幅简化计算,最后再考虑正常方法
\begin{enumerate}
\item 乘积数列极限转为定积分\\
数列通项为$n$项之积时,不能直接将数列极限转换为定积分求解,但此时可以通过取对数的方法将极限转换为级数求解,此时便可以用级数方法或者转换为定积分的方法求解了\\
例:对于极限$\lim\limits_{n\to \infty}\frac{1}{n}\sqrt[n]{\left(n+1\right)\left(n+2\right)\ldots\left(n+n\right)}$,设$a_n=\frac{1}{n}\sqrt[n]{\left(n+1\right)\left(n+2\right)\ldots\left(n+n\right)}$,并在两端取对数可得:
\[\begin{aligned}
\ln a_n&=\ln\frac{1}{n}+\frac{1}{n}\left[\ln\left(n+1\right)+\ln\left(n+2\right)+\ldots+\ln\left(n+n\right)\right]\\
&=\frac{1}{n}\left[\ln\left(n+1\right)+\ln\left(n+2\right)+\ldots+\ln\left(n+n\right)-n\ln n\right]\\
&=\frac{1}{n}\left[\ln\left(n+1\right)-\ln n+\ln\left(n+2\right)-\ln n+\ldots+\ln\left(n+n\right)-\ln n\right]\\
&=\frac{1}{n}\sum\limits_{i=1}^n\ln\frac{n+i}{n}\\
\end{aligned}\]
故我们有:
\[\lim\limits_{n\to\infty}\ln a_n=\lim\limits_{n\to\infty}\frac{1}{n}\sum\limits_{i=1}^n\ln\frac{n+i}{n}=\int_0^1\ln\left(x+1\right)dx=\ln\frac{4}{e}\]
即$\lim\limits_{n\to\infty}a_n=\frac{4}{e}$
\item 利用对称公式化简积分\\
\normalsize 设$f\left(x\right)$在区间$\left[a,b\right]$连续,则
\[\int_a^bf\left(x\right)dx=\int_a^bf\left(a+b-x\right)dx\]
例:对于积分$\int_0^1\frac{xdx}{e^x+e^{1-x}}$,设$f\left(x\right)=\frac{x}{e^x+e^{1-x}}$,则原式可化为:
\[\begin{aligned}
\int_0^1\frac{xdx}{e^x+e^{1-x}}&=\frac{1}{2}\int_0^1\left(f\left(x\right)+f\left(1-x\right)\right)dx\\
&=\frac{1}{2}\int_0^1\left(\frac{x}{e^x+e^{1-x}}+\frac{1-x}{e^{1-x}+e^{1-\left(1-x\right)}}\right)dx\\
&=\frac{1}{2}\int_0^1\frac{de^x}{e^{2x}+e}\\
&=\left.\frac{1}{2\sqrt{e}}\arctan\frac{e^x}{\sqrt{e}}\right|_0^1
\end{aligned}\]
\item 求变限积分函数$\int_{\psi\left(x\right)}^{\varphi\left(x\right)}f\left(t,x\right)dt$的导数\\
需要注意的是此时不能直接使用$\int_{\psi\left(x\right)}^{\varphi\left(x\right)}f\left(t\right)dt$的求导公式,因为函数中掺入了求导变量$x$,此时有两种思路,第一种是想办法将$f\left(t,x\right)$中的求导变量$x$提到积分号外,然后再使用$\int_{\psi\left(x\right)}^{\varphi\left(x\right)}f\left(t\right)dt$的求导公式求导;另一种办法是做变量带换,将原关于$t$的积分化为关于新变量$u$的积分$\int_{h\left(x\right)}^{g\left(x\right)}f\left(u\right)du$,然后再使用$\int_{\psi\left(x\right)}^{\varphi\left(x\right)}f\left(t\right)dt$的求导公式求导\\
例:对于式子$\frac{d}{dx}\int_{0}^{x}tf\left(x^2-t^2\right)dt$,我们设$u=x^2-t^2$,则原积分化为:
\[\int_{0}^{x}tf\left(x^2-t^2\right)dt\xlongequal{u=x^2-t^2}-\frac{1}{2}\int_{x^2}^{0}f\left(u\right)du=\frac{1}{2}\int_{0}^{x^2}f\left(u\right)du\]
故原式可得:
\[\frac{d}{dx}\int_{0}^{x}tf\left(x^2-t^2\right)dt=\frac{1}{2}\frac{d}{dx}\int_{0}^{x^2}f\left(u\right)du=xf\left(x^2\right)\]
\item 形如$\int_{a}^{a+1}f\left[g\left(x\right)\right]dx\leq\left(or\ge\right)f\left[\int_{a}^{a+1}g\left(x\right)dx\right]$的不等式的证明\\
\normalsize 先说结论,再说证明\\
结论:设$g\left(x\right)$在$x\in\left[a,a+1\right]$上的最小值和最大值均存在且分别记为$g_{min}$与$g_{max}$,若当$x\in\left[g_{min},g_{max}\right]$时我们有$f^{\prime\prime}\left(x\right)\leq 0$(或$f^{\prime\prime}\left(x\right)\ge 0$),不等式
\[\int_{a}^{a+1}f\left[g\left(x\right)\right]dx\leq f\left[\int_{a}^{a+1}g\left(x\right)dx\right]\]
\[\text{(或}\int_{a}^{a+1}f\left[g\left(x\right)\right]dx\ge f\left[\int_{a}^{a+1}g\left(x\right)dx\right]\text{)}\]
成立\\
证明:仅谈当$x\in\left[g_{min},g_{max}\right]$时$f^{\prime\prime}\left(x\right)\leq 0$的情况,另一种情况同理可证,先将$f\left(x\right)$在$x_0=\int_{a}^{a+1}g\left(x\right)dx$处泰勒展开可得:
\[f\left(x\right)=f\left(x_0\right)+f^{\prime}\left(x_0\right)\left(x-x_0\right)+\frac{f^{\prime\prime}\left(\xi\right)}{2!}\left(x-x_0\right)^2\]
其中$\xi$介于$x$与$x_0$之间,现在取$x=g\left(t\right)$,则当$t\in\left[a,a+1\right]$时显然我们有
\[f\left(g\left(t\right)\right)\leq f\left(x_0\right)+f^{\prime}\left(x_0\right)\left(g\left(t\right)-x_0\right)\]
对两端积分可得:
\[\begin{aligned}
\int_{a}^{a+1}f\left(g\left(t\right)\right)dt&\leq\int_{a}^{a+1}\left(f\left(x_0\right)+f^{\prime}\left(x_0\right)\left(g\left(t\right)-x_0\right)\right)dt\\
&=f\left(x_0\right)+f^{\prime}\left(x_0\right)\int_{a}^{a+1}\left(g\left(t\right)-x_0\right)dt\\
&=f\left(x_0\right)+f^{\prime}\left(x_0\right)\left(\int_{a}^{a+1}g\left(t\right)dt-x_0\int_{a}^{a+1}1dt\right)\\
&=f\left(x_0\right)+f^{\prime}\left(x_0\right)\left(\int_{a}^{a+1}g\left(t\right)dt-\int_{a}^{a+1}g\left(t\right)dt\int_{a}^{a+1}1dt\right)\\
&=f\left(\int_{a}^{a+1}g\left(t\right)dt\right)
\end{aligned}\]
命题得证
\item 运用留数定理计算几种特殊实定积分\\
先补充几个概念:
\begin{enumerate}
\item \textbf{极点}:对于一个复变函数$f\left(z\right)$,若它有一个孤立奇点$a$,则$a$为$f\left(z\right)$的$m$阶极点当且仅当点$a$为函数$\frac{1}{f\left(z\right)}$的$m$阶零点\\
\item \textbf{留数}:设函数$f\left(z\right)$以有限点$a$为孤立奇点,即$f\left(z\right)$在点$a$的某去心邻域$0<\left|z-a\right|<R$内解析,则称积分
\[\frac{1}{2\pi\operatorname{i}}\int_{\Gamma}f\left(z\right)dz\left(\Gamma:\left|z-a\right|=\rho,0<\rho<R\right)\]
为$f\left(z\right)$在点$a$的留数,记为$\mathop{\operatorname{Res}}\limits_{z=a}f\left(z\right)$\\
进一步地,利用复变函数在点$a$的洛朗展开式$f\left(z\right)=\sum\limits_{n=-\infty}^{+\infty}c_n\left(z-a\right)^n$(其中系数$c_n=\frac{1}{2\pi\operatorname{i}}\int_{\Gamma}\frac{f\left(\zeta\right)}{\left(\zeta-a\right)^{n+1}}d\zeta,n=0,\pm1,\pm2,\cdots,\Gamma:\left|z-a\right|=\rho,0<\rho<R$)可以得到,展式中的$n=-1$项的系数
\[c_{-1}=\frac{1}{2\pi\operatorname{i}}\int_{\Gamma}\frac{f\left(\zeta\right)}{\left(\zeta-a\right)^{-1+1}}d\zeta=\frac{1}{2\pi\operatorname{i}}\int_{\Gamma}f\left(z\right)dz=\mathop{\operatorname{Res}}\limits_{z=a}f\left(z\right)\]
即为函数在该点的留数,即$f\left(z\right)$在点$a$处的洛朗展式中$\frac{1}{z-a}$这一项的系数即为函数在该点的留数\\
\item \textbf{留数计算方法}:设$a$为$f\left(z\right)$的$n$阶极点,此时$f\left(z\right)$可表示为:
\[f\left(z\right)=\frac{\varphi\left(z\right)}{\left(z-a\right)^n}\]
其中$\varphi\left(z\right)$在点$a$处解析,$\varphi\left(z\right)\ne0$,则我们有:
\[\mathop{\operatorname{Res}}\limits_{z=a}f\left(z\right)=\frac{\varphi^{\left(n-1\right)}\left(a\right)}{\left(n-1\right)!}\]
这里符号$\varphi^{\left(0\right)}\left(a\right)$表示$\varphi\left(a\right)$,$\varphi^{\left(n-1\right)}\left(a\right)$表示$\lim\limits_{z\to a}\varphi^{\left(n-1\right)}\left(z\right)$\\
\item \textbf{柯西留数定理}:复变函数$f\left(z\right)$在周线或复周线$C$上所围的区域$D$内,除点$a_1,a_2,\cdots,a_n$外解析,在闭域$\bar{D}=D+C$上除点$a_1,a_2,\cdots,a_n$外连续,则我们有:
\[\int_{C}f\left(z\right)dz=2\pi\operatorname{i}\sum\limits_{k=1}^{n}\mathop{\operatorname{Res}}\limits_{z=a_k}f\left(z\right)\]
\end{enumerate}
几种实积分的计算方法:
\begin{enumerate}
\item $\int_{a}^{2\pi+a}R\left(\cos\theta,\sin\theta\right)d\theta$型积分\\
设$z=e^{\operatorname{i}\theta}=\cos\theta+\operatorname{i}\sin\theta$,则:
\[\cos\theta=\frac{z+z^{-1}}{2},\sin\theta=\frac{z-z^{-1}}{2\operatorname{i}},d\theta=\frac{1}{\operatorname{i}z}dz\]
此时积分计算可以变为:
\[\int_{a}^{2\pi+a}R\left(\cos\theta,\sin\theta\right)d\theta=\int_{\left|z\right|=1}R\left(\frac{z+z^{-1}}{2},\frac{z-z^{-1}}{2\operatorname{i}}\right)\frac{dz}{\operatorname{i}z}\]
此时再使用留数定理即可求解。此外需要注意的是,有时我们可以结合上面提到过的三角函数的一些特殊积分公式将非$\int_{a}^{2\pi+a}R\left(\cos\theta,\sin\theta\right)d\theta$型的积分化为这种形式的积分,再使用留数方法求解,例如当$R\left(\cos\theta,\sin\theta\right)$为$\theta$的偶函数时,有
\[\int_{0}^{\pi}R\left(\cos\theta,\sin\theta\right)d\theta=\frac{1}{2}\int_{-\pi}^{\pi}R\left(\cos\theta,\sin\theta\right)d\theta\]
\item $\int_{-\infty}^{+\infty}\frac{P\left(x\right)}{Q\left(x\right)}dx$型积分\\
这种类型的积分需要使用到一个定理:\\
定理:设$f\left(z\right)=\frac{P\left(z\right)}{Q\left(z\right)}$为有理分式,其中:
\[\begin{aligned}
P\left(z\right)&=c_0z^m+c_1z^{m-1}+\cdots+c_m,c_0\ne0\\
Q\left(z\right)&=b_0z^n+b_1z^{n-1}+\cdots+b_n,b_0\ne0
\end{aligned}\]
且$P\left(z\right)$与$Q\left(z\right)$为互质多项式,此外还符合条件:
\begin{enumerate}
\item $n-m\ge2$
\item 在实轴上$Q\left(z\right)\ne0$
\end{enumerate}
则我们有:
\[\int_{-\infty}^{+\infty}f\left(x\right)dx=2\pi\operatorname{i}\sum\limits_{\operatorname{Im}a_k>0}\mathop{\operatorname{Res}}\limits_{z=a_k}f\left(z\right)\]
\item $\int_{-\infty}^{+\infty}\frac{P\left(x\right)}{Q\left(x\right)}\cos mxdx$或$\int_{-\infty}^{+\infty}\frac{P\left(x\right)}{Q\left(x\right)}\sin mxdx$型积分\\
实际上这两个积分都可以视为$\int_{-\infty}^{+\infty}\frac{P\left(x\right)}{Q\left(x\right)}e^{\operatorname{i}mx}dx$型积分问题来求解,在求出$\int_{-\infty}^{+\infty}\frac{P\left(x\right)}{Q\left(x\right)}e^{\operatorname{i}mx}dx$后,由于
\[\begin{aligned}
&\int_{-\infty}^{+\infty}\frac{P\left(x\right)}{Q\left(x\right)}e^{\operatorname{i}mx}dx\\
=&\int_{-\infty}^{+\infty}\frac{P\left(x\right)}{Q\left(x\right)}\left(\cos mx+\operatorname{i}\sin mx\right)dx\\
=&\int_{-\infty}^{+\infty}\frac{P\left(x\right)}{Q\left(x\right)}\cos mxdx+\operatorname{i}\int_{-\infty}^{+\infty}\frac{P\left(x\right)}{Q\left(x\right)}\sin mxdx
\end{aligned}\]
故由$\int_{-\infty}^{+\infty}\frac{P\left(x\right)}{Q\left(x\right)}e^{\operatorname{i}mx}dx$的实部和虚部可直接得到$\int_{-\infty}^{+\infty}\frac{P\left(x\right)}{Q\left(x\right)}\cos mxdx$和$\int_{-\infty}^{+\infty}\frac{P\left(x\right)}{Q\left(x\right)}\sin mxdx$的值,即:
\[\begin{aligned}
&\int_{-\infty}^{+\infty}\frac{P\left(x\right)}{Q\left(x\right)}\cos mxdx=\operatorname{Re}\left[\int_{-\infty}^{+\infty}\frac{P\left(x\right)}{Q\left(x\right)}e^{\operatorname{i}mx}dx\right],\\
&\int_{-\infty}^{+\infty}\frac{P\left(x\right)}{Q\left(x\right)}\sin mxdx=\operatorname{Im}\left[\int_{-\infty}^{+\infty}\frac{P\left(x\right)}{Q\left(x\right)}e^{\operatorname{i}mx}dx\right]
\end{aligned}\]
而要计算积分$\int_{-\infty}^{+\infty}\frac{P\left(x\right)}{Q\left(x\right)}e^{\operatorname{i}mx}dx$,我们需要如下定理:\\
定理:设$g\left(z\right)=\frac{P\left(z\right)}{Q\left(z\right)}$为有理分式,其中$P\left(z\right)$与$Q\left(z\right)$为互质多项式,且符合条件:
\begin{enumerate}
\item $Q\left(z\right)$的次数比$P\left(z\right)$的次数高
\item 在实轴上$Q\left(z\right)\ne0$
\item $m>0$
\end{enumerate}
则我们有:
\[\int_{-\infty}^{+\infty}g\left(x\right)e^{\operatorname{i}mx}dx=2\pi\operatorname{i}\sum\limits_{\operatorname{Im}a_k>0}\mathop{\operatorname{Res}}\limits_{z=a_k}\left[g\left(z\right)e^{\operatorname{i}mz}\right]\]
\end{enumerate}
\end{enumerate}


\section{解析几何}
\begin{enumerate}
\item 向量有关证明问题\\
可以选取两个(平面问题)或者三个(空间问题)不共线的向量作为基向量,然后将问题中出现的其他向量全部用这一组基向量的线性组合表示,这样通常情况下可以将问题简化(计算量可能不会简化)
\item 求解与两直线分别共面或者平行的平面\\
求解与两直线分别共面或者平行的平面时可以考虑三向量共面的方法求解,将待求平面上的点与已知平面要过的一个点可以表示出平面上的向量$\vec{v}=\left(x-x_0,y-y_0,z-z_0\right)$,然后将这个可以表示平面上所有向量的向量$\vec{v}$带入三向量共面的方程即可解出平面方程,列出三向量共面的方程通常可以使用两种方法,一是通过两向量可以线性表出第三个向量列出方程,二是通过三向量混合积为零列出方程\\
例:已知两直线方程为$l_1:\frac{x-1}{1}=\frac{y-2}{0}=\frac{z-3}{-1}$和$l_2:\frac{x+2}{2}=\frac{y-1}{1}=\frac{z}{1}$,对于过直线$l_1$且与$l_2$的平面方程,易知平面过点$\left(1,2,3\right)$,则平面上的向量可以表示为$\vec{v}=\left(x-1,y-2,z-3\right)$,现在平面方程就可以通过三向量共面混合积为零求解:
\[\begin{vmatrix}x-1&y-2&z-3\\1&0&-1\\2&1&1\end{vmatrix}=0\Rightarrow x-3y+z+2=0\]
\item 三平面的位置关系问题\\
三平面的位置关系问题可以化为三阶线性方程组的解的存在性以及数量问题,此时可以利用线性代数里的知识求解
\item 对称点和对称直线求解\\
可以利用中点坐标公式$x_2=\frac{x_1+x_3}{2}$,$y_2=\frac{y_1+y_3}{2}$,$z_2=\frac{z_1+z_3}{2}$,在求出中点后再结合原本的点就能得到对称点,而对称直线则可以通过求直线上的任意两点的对称点求出对称直线,对称平面同理
\end{enumerate}


\section{重积分}
\begin{enumerate}
\item 利用重积分换元法简化计算\\
重积分换元法是非常好用且灵活的方法,遇到通常方法无法积分的重积分时,可以考虑观察重积分的形式,然后选取合适的变量代换来简化计算\\
例:要计算二重积分在平行四边形区域$D$内的二重积分$\iint\limits_{D}\left(x-y\right)^2\sin^2\left(x+y\right)dxdy$,其中$D$的四个顶点位置为$\left(\pi,0\right)$、$\left(2\pi,\pi\right)$、$\left(\pi,2\pi\right)$和$\left(0,\pi\right)$,此时我们可以利用二重积分换元法来简化计算,设$u=x-y$、$v=x+y$,则我们有$x=\frac{u+v}{2}$、$y=\frac{v-u}{2}$,此时原积分可化为:
\[\begin{aligned}
&\iint\limits_{D}\left(x-y\right)^2\sin^2\left(x+y\right)dxdy\\
=&\iint\limits_{D^{\prime}}u^2\sin^2v\cdot\left|\frac{\partial\left(x,y\right)}{\partial\left(u,v\right)}\right|dudv\\
=&\frac{1}{2}\int_{-\pi}^{\pi}u^2du\int_{\pi}^{3\pi}\sin^2vdv\\
=&\frac{\pi^4}{3}
\end{aligned}\]
\item 单次定积分化为重积分简化计算\\
对于一个单次定积分,当被积函数为一个不方便积分的函数时,可以考虑将这个被积函数再化为一个单次积分的形式,即将整个单次定积分转换为一个重积分,然后再利用交换积分次序等重积分求解方法将其求解。类似的,对于二重积分也可以考虑转换为三重积分计算(这种情况较少)\\
例1:
\[\begin{aligned}
&\int_{0}^{1}\frac{\ln\left(1+x\right)}{\left(2-x\right)^2}dx\\
=&\int_{0}^{1}\frac{dx}{\left(2-x\right)^2}\int_{0}^{x}\frac{dy}{1+y}\\
=&\iint\limits_{D}\frac{1}{\left(1+y\right)\left(2-x\right)^2}dxdy\\
=&\int_{0}^{1}\frac{dy}{1+y}\int_{y}^{1}\frac{dx}{\left(2-x\right)^2}\\
=&\frac{1}{3}\ln2
\end{aligned}\]
例2:
\[\begin{aligned}
&\int_{0}^{+\infty}e^{-x^2}dx\\
=&\sqrt{\left(\int_{0}^{+\infty}e^{-x^2}dx\right)^2}\\
=&\sqrt{\int_{0}^{+\infty}e^{-x^2}dx\cdot\int_{0}^{+\infty}e^{-y^2}dy}\\
=&\sqrt{\int_{0}^{+\infty}\int_{0}^{+\infty}e^{-\left(x^2+y^2\right)}dxdy}\\
=&\sqrt{\int_{0}^{\frac{\pi}{2}}d\theta\int_{0}^{+\infty}e^{-r^2}rdr}\\
=&\frac{\sqrt{\pi}}{2}
\end{aligned}\]
\item 利用几何面积求解积分(尤其是圆的面积)\\
当积分不易求解但其几何含义表示的是某个易求的图形面积时可以考虑直接求面积代替求积分,最常见的就是圆的面积,例如形如
\[\int_{0}^{a}\sqrt{a^2-x^2}dx\]
的积分均可看为求解圆心为原点半径为$a$的第一象限的扇形面积\\
例:
\[\begin{aligned}
&\int_{0}^{1}dy\int_{0}^{1}\sqrt{e^{2x}-y^2}dx+\int_{1}^{e}dy\int_{\ln y}^{1}\sqrt{e^{2x}-y^2}dx\\
=&\int_{0}^{1}dx\int_{0}^{e^x}\sqrt{e^{2x}-y^2}dy\\
=&\int_{0}^{1}\frac{1}{4}\pi\left(e^x\right)^2dx\\
=&\frac{\pi}{8}\left(e^2-1\right)
\end{aligned}\]
\end{enumerate}


\section{曲线积分与曲面积分}
\begin{enumerate}
\item 一种特殊的第一类曲线积分\\
设曲线$L$的外法线向量为$\vec{n}$,则对于函数$f\left(x,y\right)$,我们有这样的曲线积分公式:
\[\int_{L}\frac{\partial f}{\partial\vec{n}}ds=\int_{L}\frac{\partial f}{\partial x}dy-\frac{\partial f}{\partial y}dx\]
此时就将问题转换为了第二类曲线积分的求解
\end{enumerate}


\section{级数}
\begin{enumerate}
\item 级数单个通项也是形如级数的级数敛散性证明(该方法不能用于求级数和的具体值)\\
若级数$\sum\limits_{n=1}^{\infty}a_n$的单个通项$a_n$也是形如级数的形式,则可以利用几种级数审敛法将级数$\sum\limits_{n=1}^{\infty}a_n$敛散性的证明转化为与单个通项$a_n$有关的级数敛散性的证明\\
例:要证明级数$\sum\limits_{n=1}^{\infty}\frac{n}{1^p+2^p+\cdots+n^p}(p>1)$收敛,可以利用比较审敛法的极限形式,将其转换为证明$\lim\limits_{n\to\infty}\frac{a_n}{b_n}$极限存在,由于$a_n=\frac{n}{1^p+2^p+\cdots+n^p}$形如级数,故我们选择$b_n=\frac{1}{n^p}$可以将证明$\lim\limits_{n\to\infty}\frac{a_n}{b_n}$极限存在转换为证明级数和存在,即我们有:
\[\begin{aligned}
\lim\limits_{n\to\infty}\frac{a_n}{b_n}&=\lim\limits_{n\to\infty}\frac{\frac{n}{1^p+2^p+\cdots+n^p}}{\frac{1}{n^p}}\\
&=\frac{1}{\lim\limits_{n\to\infty}\frac{1}{n}\left[\left(\frac{1}{n}\right)^p+\left(\frac{2}{n}\right)^p+\cdots+\left(\frac{n}{n}\right)^p\right]}\\
&=\frac{1}{\int_{0}^{1}x^pdx}\\
&=p+1
\end{aligned}\]
即我们知道极限存在,由于$\sum\limits_{n=1}^{\infty}b_n=\sum\limits_{n=1}^{\infty}\frac{1}{n^p}$在$p>1$时收敛,故原级数$\sum\limits_{n=1}^{\infty}a_n=\sum\limits_{n=1}^{\infty}\frac{n}{1^p+2^p+\cdots+n^p}$收敛
\end{enumerate}





\end{document}















































